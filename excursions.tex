\documentclass[a4paper]{scrartcl}
\usepackage{unicode-math}
\usepackage{luamplib}
% turn on automatic TeX labels
\mplibtextextlabel{enable}
% add at the start/end of each fig
\everymplib{input colorbrewer-rgb;interim ahangle := 30;beginfig(0);}
\everyendmplib{endfig;}
%
%\usepackage{shortvrb}\MakeShortVerb{"}
\usepackage{mflogo}
\title{Excursions in \MP}
\author{Toby Thurston}
\date{October 2022 —}
\begin{document}
\maketitle

\noindent
This example document includes geometric illustrations inspired by
\textsl{Excursions in Geometry}, C.\@ Stanley Ogilvy, OUP 1968.
The illustrations are presented roughly in the same order as the book, with notes
about how you can use \MP\ to produce similar.  The section heading also
approximately follow the book.  You might like to read the PDF of
this document side by side with the source code, so that you can see how each
illustration is done.  Each illustration is included as in-line \MP\ code, there are
no external graphics files used.

\section{A bit of background}

Ogilvie starts with a review of some circle theorems.  In fact most of the book is
about circles in one way or another.

In this first diagram, you are given the width $AB$ of the screen and the ideal viewing angle $\theta$.
The \MP\ code works out the rest from that, including a useful routine for a circle through three points.
$$
\begin{mplibcode}
numeric theta;
pair A, B, V, U;
path c;

% given: width of the screen and ideal viewing angle
A = -B = 60 left;
theta = 42;  

% use sine rule to find V
V = sind(90 - 1/2 theta) / sind(theta) * abs (A-B) * dir (1/2 theta - 90) shifted A;

vardef circle_through(expr a, b, c) = 
   save m; pair m; m = whatever * (a-b) rotated 90 shifted 1/2[a,b]
                     = whatever * (a-c) rotated 90 shifted 1/2[a,c];
   fullcircle scaled 2 abs (a-m) shifted m
enddef;

c = circle_through(A, B, V);

U = point 4.8 of c;

draw A -- V -- B -- U -- A;
draw A--B withpen pencircle scaled 1;

clip currentpicture to c; % clip the thick pen
draw c withcolor Blues 8 4;

label.top("\small Screen", 1/2[A, B]);
forsuffixes $ = U, V:
    label("$\theta$", $ + 13 * dir (1/2 theta + angle (B - $))) withcolor Reds 8 7;
endfor
forsuffixes $ = A, B, U, V:
    label(str $, $ shifted - center c scaled 1.08 shifted center c); 
endfor
\end{mplibcode}
$$
Here $\theta$ is half the angle measured by the intercepted arc, which gives us the useful corollary that 
any angle inscribed in a semicircle is a right angle, and conversely that if you can show that some angle $ACB$
    is a right angle, then the semicircle drawn with $AB$ as a diameter must pass through $C$.
$$
\begin{mplibcode}
pair A, B, C; C = origin; B = 150 dir -50; A = 80 dir -140;
draw halfcircle scaled abs (B-A) rotated angle (B-A) shifted 1/2[B, A] withcolor Blues 8 4;
draw A -- B -- C -- cycle;
forsuffixes $ = A, B, C:
    label(str $, $ shifted - 1/2[A,B] scaled 1.08 shifted 1/2[A,B]); 
endfor
\end{mplibcode}
$$

\end{document}
