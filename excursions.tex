\documentclass[a4paper]{scrartcl}
\usepackage{unicode-math}
\usepackage{luamplib}
\usepackage{dwmpcode}
% turn on automatic TeX labels
\mplibtextextlabel{enable}
% add at the start/end of each fig
\everymplib{input colorbrewer-rgb;interim ahangle := 30;beginfig(0);}
\everyendmplib{endfig;}
%
%\usepackage{shortvrb}\MakeShortVerb{"}
\usepackage{mflogo}
\title{Excursions in \MP}
\author{Toby Thurston}
\date{October 2022 —}
\begin{document}
\maketitle

\noindent
This example document includes geometric illustrations inspired by
\textsl{Excursions in Geometry}, C.\@ Stanley Ogilvy, OUP 1968.
The illustrations are presented roughly in the same order as the book, with notes
about how you can use \MP\ to produce similar.  The section heading also
approximately follow the book.  You might like to read the PDF of
this document side by side with the source code, so that you can see how each
illustration is done.  Each illustration is included as in-line \MP\ code, there are
no external graphics files used.

\section{A bit of background}

Ogilvie starts with a review of some circle theorems.  In fact most of the book is
about circles in one way or another.

In this first diagram, you are given the width $AB$ of the screen and the ideal viewing angle $\theta$.
The \MP\ code works out the rest from that, including a useful routine for a circle through three points.
$$
\begin{mplibcode}
numeric theta;
pair A, B, V, U;
path c;

% given: width of the screen and ideal viewing angle
A = -B = 60 left;
theta = 42;  

% use sine rule to find V
V = sind(90 - 1/2 theta) / sind(theta) * abs (A-B) * dir (1/2 theta - 90) shifted A;

vardef circle_through(expr a, b, c) = 
   save m; pair m; m = whatever * (a-b) rotated 90 shifted 1/2[a,b]
                     = whatever * (a-c) rotated 90 shifted 1/2[a,c];
   fullcircle scaled 2 abs (a-m) shifted m
enddef;

c = circle_through(A, B, V);

U = point 4.8 of c;

draw A -- V -- B -- U -- A;
draw A--B withpen pencircle scaled 1;

clip currentpicture to c; % clip the thick pen
draw c withcolor Blues 8 4;

label.top("\small Screen", 1/2[A, B]);
forsuffixes $ = U, V:
    label("$\theta$", $ + 13 * dir (1/2 theta + angle (B - $))) withcolor Reds 8 7;
endfor
forsuffixes $ = A, B, U, V:
    label(str $, $ shifted - center c scaled 1.08 shifted center c); 
endfor
\end{mplibcode}
$$
Here $\theta$ is half the angle measured by the intercepted arc, which gives us the useful corollary that 
any angle inscribed in a semicircle is a right angle, and conversely that if you can show that some angle $ACB$
    is a right angle, then the semicircle drawn with $AB$ as a diameter must pass through $C$.
$$
\begin{mplibcode}
pair A, B, C; C = origin; B = 150 dir -50; A = 80 dir -140;
% you can do a "semicircle drawn with AB as a diameter" like this
draw halfcircle scaled abs (B-A) 
                rotated angle (B-A) 
                shifted 1/2[B, A] withcolor Blues 8 4;
draw A -- B -- C -- cycle;
forsuffixes $ = A, B, C:
    label(str $, $ shifted - 1/2[A,B] scaled 1.08 shifted 1/2[A,B]); 
endfor
\end{mplibcode}
$$

\smallskip\noindent
A basic theorem: If two chords intersect, the product of the lengths of the segments of 
the one equals the product of the lengths of the segments of the other. In all three cases
below you have $PA/PD = PC/PB$ by similar triangles, hence $PA\cdot PB=PC\cdot PD$.
$$
\begin{mplibcode}
path base; base = fullcircle scaled 120;
picture P[];
P1 = image(
    numeric d; d = 0.2;
    pair A, B, C, D, P;
    A = point 5 of base;
    B = point 2.5 of base;
    C = point 4 + d of base;
    D = point 8 - d of base;

    P = whatever[A, B] = whatever[C, D];

    draw C -- A -- D -- B dashed evenly scaled 1/2 
        withpen pencircle scaled 1/4
        withcolor 3/4;
    draw C -- D;
    draw A -- B;

    forsuffixes $ = A, B, C, D:
        label("$" & str $ & "$", $ scaled 1.1); 
    endfor
    label.lrt("$P$", P);

    drawoptions(withcolor Reds 7 6);
    label("$\scriptscriptstyle\alpha$", B shifted 4 (unitvector(A-B) + unitvector (D-B)));
    label("$\scriptscriptstyle\alpha$", C shifted 4 (unitvector(A-C) + unitvector (D-C)));
    label("$\scriptscriptstyle\beta$",  P shifted 4 (unitvector(A-P) + unitvector (C-P)));
    label("$\scriptscriptstyle\beta$",  P shifted 4 (unitvector(B-P) + unitvector (D-P)));
    drawoptions();
    draw base withcolor Blues 8 4;
    label.urt("I.", lrcorner currentpicture) withcolor Purples 7 7;
);
P2 = image(
    numeric d; d = 0.9;
    pair A, B, C, D, P;
    A = point 3.8 of base;
    B = point 2 of base;
    C = point 4 + d of base;
    D = point 8 - d of base;

    P = whatever[A, B] = whatever[C, D];

    draw B -- (1 + 12 / abs(B-P))[B, P];
    draw D -- (1 + 12 / abs(D-P))[D, P];

    label.ulft("$A$", A);
    label.top ("$B$", B);
    label.llft("$C$", C);
    label.lrt ("$D$", D);
    label.ulft("$P$", P);

    draw base withcolor Blues 8 4;
    label.urt("II.", lrcorner currentpicture) withcolor Purples 7 7;
);
P3 = image(
    numeric d, t; d = 0.7; t = 2.6;
    pair A, B, C, D, P;
    A = B = point t of base;
    C = point 4 + d of base;
    D = point 8 - d of base;

    P = whatever[B, B shifted direction t of base] = whatever[C, D];

    draw B -- (1 + 14 / abs(B-P))[B, P];
    draw D -- (1 + 14 / abs(D-P))[D, P];

    label.ulft("$A,B$", A);
    label.llft("$C$", C);
    label.lrt ("$D$", D);
    label.ulft("$P$", P);
    draw base withcolor Blues 8 4;
    label.urt("III.", lrcorner currentpicture) withcolor Purples 7 7;
);

interim labeloffset := 34;
label.ulft(P1, origin);
label.urt (P2, origin);
label.bot (P3, origin);
\end{mplibcode}
$$
\centerline{This example also shows how to arrange sub-figures into one.}
\clearpage
\section{Harmonic division and Apollonian circles}

Can we find $C$ and $D$ on the line $AB$ so that $AC/CB = AD/BD$? 

\medskip\noindent
Yes:
$$
\begin{mplibcode}
pair A, B, C, D;
A = origin;
numeric a, b, c;
a = 150; b = 50; c * a / b = a + b + c;
B = (a+b, 0);
C = (a, 0);
D = (a+b+c, 0);

draw A--D;
forsuffixes $ = A, B, C, D:
    draw (up--down) scaled 2 shifted $;
    label.bot("$" & str $ & "$", $);
endfor
\end{mplibcode}
$$

\bigskip\noindent
\textit{Theorem}. The bisector of any angle of a triangle divides
the opposite side into parts proportional to the adjacent sides.

\smallskip\noindent
So given $P$, with $AP=k$ and $BP=1$:
$$ % Fig 8 in the book
\begin{mplibcode}
pair A, B, C, D, P;
A = origin;
numeric a, b, c, k;
a = 180; b = 50; c * a / b = a + b + c;
B = (a+b, 0);
C = (a, 0);
D = (a+b+c, 0);
k = a / b;
path aa, bb; 
aa = fullcircle scaled 144k shifted A;
bb = fullcircle scaled 144  shifted B;
P = aa intersectionpoint bb;

draw C -- P -- D dashed evenly scaled 1/2 
                 withpen pencircle scaled 1/4
                 withcolor 3/4;
draw A -- (1+26/abs(A-P))[A, P];
draw B -- P;
draw A--D;
forsuffixes $ = A, B, C, D:
    dotlabel.bot("$" & str $ & "$", $);
endfor
dotlabel.ulft("$P$", P);
label.ulft("$k$", 1/2[A, P]);
label.rt("$1$", 1/2[B, P]);
drawoptions(withcolor Reds 7 6);
label("$\scriptscriptstyle\alpha$", P shifted 8 (unitvector(A-P) + unitvector (C-P)));
label("$\scriptscriptstyle\alpha$", P shifted 8 (unitvector(C-P) + unitvector (B-P)));
label("$\scriptscriptstyle\beta$",  P shifted 6 (unitvector(B-P) + unitvector (D-P)));
label("$\scriptscriptstyle\beta$",  P shifted 6 (unitvector(P-A) + unitvector (D-P)));
drawoptions();
\end{mplibcode}
$$
We have $AC/CB = AP/BP = k$ from the interior angles and $AD/BD = AP/BP = k$ from the exterior pair.
The proof looks like this:
$$
\begin{mplibcode}
pair A, B, C, D, P, E, F, G, H;
A = origin;
numeric a, b, c, k;
a = 180; b = 50; c * a / b = a + b + c;
B = (a+b, 0);
C = (a, 0);
D = (a+b+c, 0);
k = a / b;
path aa, bb; 
aa = fullcircle scaled 144k shifted A;
bb = fullcircle scaled 144  shifted B;
P = aa intersectionpoint bb;

E = whatever[A, P]; E - B = whatever * (P - C);
F = whatever[B, E] = whatever[P, D];
G = whatever[A, P]; G - B = whatever * (P - D);
H = whatever[B, G] = whatever[P, C];

draw unitsquare scaled 5 rotated angle (E-F) shifted F 
                 withpen pencircle scaled 1/4
                 withcolor 3/4;
draw unitsquare scaled 5 rotated angle (P-H) shifted H 
                 withpen pencircle scaled 1/4
                 withcolor 3/4;
draw C -- P -- D dashed evenly scaled 1/2 
                 withpen pencircle scaled 1/4
                 withcolor 1/2;
draw G -- B -- E dashed evenly scaled 1/2 
                 withpen pencircle scaled 1/4
                 withcolor 1/2;
draw A -- (1+20/abs(A-E))[A, E];
draw B -- P;
draw A--D;
forsuffixes $ = A, B, C, D:
    dotlabel.bot("$" & str $ & "$", $);
endfor
forsuffixes $ = E, G, P:
    dotlabel.ulft("$" & str $ & "$", $);
endfor
label.rt("\ $F$",F);
drawoptions(withcolor Reds 7 6);
label("$\scriptscriptstyle\alpha$", P shifted 8 (unitvector(A-P) + unitvector (C-P)));
label("$\scriptscriptstyle\alpha$", P shifted 8 (unitvector(C-P) + unitvector (B-P)));
label("$\scriptscriptstyle\beta$",  P shifted 6 (unitvector(B-P) + unitvector (D-P)));
label("$\scriptscriptstyle\beta$",  P shifted 6 (unitvector(P-A) + unitvector (D-P)));
drawoptions();
\end{mplibcode}
$$
Since the lines $PC$ and $PD$ are bisectors, then we have $2\alpha + 2\beta = 180°$,
hence $\alpha+\beta$ is a right angle, so if we draw $BE$ parallel to $PC$ it will
be perpendicular to $PD$, and hence the two triangles $PFE$ and $PFB$ are congruent,
so that $PE=PB$.  But the parallel lines cut off proportional segments, so that 
$$
\frac{AC}{CB} = \frac{AP}{PE} = \frac{AP}{BP}
$$
as required.  And using the exterior angle
$$
\frac{AD}{BD} = \frac{AP}{GP} = \frac{AP}{BP}
$$

\clearpage
\subsection{Applied Apollonian example}

Given $A$, $B$, $k$, and course of the target, find intersection point $Q$.
$$
\begin{mplibcode}
pair A, B, C, D, P, Q;
A = origin; B = 180 right; k = 2;
C = (k/(k+1))[A, B];
D = (k/(k-1))[A, B];
path app; 
app = fullcircle scaled length (C-D) shifted 1/2[C,D];
P = point 2.236 of app;
Q = app intersectionpoint (B -- B + 1024 dir -60);
%Q = point 6 of app;
draw app withcolor 3/4[blue, white];
draw A -- D -- P -- cycle;
draw B -- P -- C;
numeric r; r = 3/8;
draw r[B, Q] -- Q -- r[A, Q] withpen pencircle scaled 1/4 dashed evenly;
drawarrow B -- r[B, Q];
drawarrow A -- r[A, Q];
% label.ulft("$" & decimal (length (P-A) / length (P-B)) & "$", 1/2[A, P]);
% numeric alpha; alpha = angle (Q-A);
% label(decimal alpha, A + 30 dir 1/2 alpha);
label.lft ("$A$", A);
label.ulft("$B$", B);
label.ulft("$C$", C);
label.rt  ("$D$", D);
label.top ("$P$", P);
label.lrt ("$Q$", Q);
    drawoptions(withcolor 3/4 red);
       label.top("$a$", 1/2[A,C]); 
       label.top("$b$", 1/2[B,C]); 
       label.top("$c$", 1/2[B,D]); 
    drawoptions();
\end{mplibcode}
$$
But how can you find $C$ and $D$ in \MP\ given $A$, $B$, and $k$?  Using
\textcolor{textred}{$a$}, 
\textcolor{textred}{$b$}, and 
\textcolor{textred}{$c$} 
for the lengths as shown above, then using the mediation syntax,
point $C$ is \mpl{(a/(a+b))[A, B]} and $D$ is \mpl{((a+b+c)/(a+b))[A, B]}, so what
are these fractions in terms of $k$?
We know that $AC/CB=AD/BD=k$ so we have 
$a/b = (a+b+c)/c = k$ which we can develop as follows.

Starting with $a/b=k$, add 1 to each side, $a/b + b/b = k + 1$, hence $(a+b)/b = k +
1$.  We can then divide the first equation by the last to get $a/(a+b) = k/(k+1)$.

Similarly, starting with $(a+b+c)/c = k$, take 1 away from each side to get
$(a+b+c)/c - c/c = k - 1$, hence $(a+b)/c = k-1$.  And again dividing the first by
the last gives $(a+b+c)/(a+b) = k / (k-1)$. 

\smallskip\noindent
So our \MP\ expressions for points $C$ and $D$ can be written like this:
\begin{code}
    C = (k/(k+1))[A, B]; D = (k/(k-1))[A, B];
\end{code}
If $k=1$ you will get a “Division by zero” error from the second expression.
This corresponds to $D$ at infinity and $C$ half way from $A$ to $B$.  The circle
through these two points can be thought of as the perpendicular bisector of $A$ and
$B$. If you have $0 < k < 1$ then the circle will be round $A$ not $B$.  If $k<0$
the positions of $C$ and $D$ will be swapped, and there is another error when
$k=-1$.

\end{document}
