\documentclass[oneside]{scrbook}
\usepackage{unicode-math}
\setmainfont{TeX Gyre Pagella}
\setmathfont{TeX Gyre Pagella Math}
\usepackage{graphicx}
\usepackage{mflogo}
\addtokomafont{section}{\clearpage}
\usepackage{luamplib}
\mplibtextextlabel{enable}
\everymplib{input colorbrewer-rgb;interim ahangle := 30;beginfig(0);}
\everyendmplib{
numeric wd, tw;
wd = xpart (urcorner bbox currentpicture - llcorner bbox currentpicture);
tw = \mpdim{\textwidth};
if wd > tw: currentpicture := currentpicture scaled (tw/wd); fi
endfig;}
\usepackage{minitoc}
\mtcsetrules{*}{off}
%--------------------
\title{Proofs without words I\\[6pt]{\Large Exercises in \MP}}
\date{March 2021 —\\[4in]
\centerline{\begin{mplibcode}
    path t[], h;
    numeric r; r = -30;
    t0 = (for i=0 upto 2: up scaled 21 rotated 120i -- endfor cycle) rotated r;
    h  = (for i=0 upto 5: up scaled 34 rotated 60i -- endfor cycle) rotated r;
    t1 = subpath (0, 1) of t0 -- point 1 of h -- cycle;
    t2 = subpath (1, 2) of t0 -- point 3 of h -- cycle;
    t3 = subpath (2, 3) of t0 -- point 5 of h -- cycle;
    t4 = subpath (0, 1) of h -- point 0 of t0 -- cycle;
    t5 = subpath (1, 2) of h -- point 1 of t0 -- cycle;
    t6 = subpath (2, 3) of h -- point 1 of t0 -- cycle;
    t7 = subpath (3, 4) of h -- point 2 of t0 -- cycle;
    t8 = subpath (4, 5) of h -- point 2 of t0 -- cycle;
    t9 = subpath (5, 6) of h -- point 0 of t0 -- cycle;
    fill t0 withcolor Blues 7 2;
    fill t1 withcolor Blues 7 1;
    fill t2 withcolor Blues 7 3;
    fill t3 withcolor Blues 7 3;
    fill t4 withcolor Blues 7 1;
    fill t5 withcolor Blues 7 1;
    fill t6 withcolor Blues 7 2;
    fill t7 withcolor Blues 7 5;
    fill t8 withcolor Blues 7 5;
    fill t9 withcolor Blues 7 2;
    forsuffixes @=0, 4, 5, 6, 7, 8, 9: draw t@; endfor
\end{mplibcode}}}
\author{Toby Thurston}
\setcounter{secnumdepth}{-1}
\setcounter{tocdepth}{0}
\def\contrib#1{\rightline{— #1}}
\def\implies{\ensuremath\enspace\Longrightarrow\enspace}
\def\sint{\sin\theta}
\def\cost{\cos\theta}

\begin{document}
\dominitoc[n]
\maketitle
\tableofcontents

\chapter{Geometry and Algebra}

\minitoc

\section{The Pythagorean theorem I}

\vfill
$$
\begin{mplibcode}
path s, t;
s = unitsquare shifted -(1/2, 1/2) scaled 144;
t = point 0 of s -- point 2/3 of s -- point -1/3 of s -- cycle;
picture P[];
P2 = image(
    for i=0 upto 3:
        fill t rotated 90i withcolor if odd i: Blues 7 2 else: Oranges 7 2 fi;
        draw t rotated 90i;
    endfor
    draw s;
);
P1 = image(
    fill t withcolor Oranges 7 2; draw t;
    t := t rotatedabout(point 3/2 of t, 180);
    fill t withcolor Oranges 7 2; draw t;
    t := t shifted (point 0 of t - point 2 of t);
    t := t rotatedabout(point 2 of t, -90);
    fill t withcolor Blues 7 2; draw t;
    t := t rotatedabout(point 3/2 of t, 180);
    fill t withcolor Blues 7 2; draw t;
    draw s;
);
draw P1;
draw P2 shifted 200 right;
\end{mplibcode}
$$
\vfill
\contrib{adapted from the \textit{Chou pei san ching}}

\section{The Pythagorean theorem II}

\vfill
$$
\begin{mplibcode}
path s, t;
s = fullcircle scaled 144;
t = (point 4 of s -- point 0 of s -- point sqrt(2) of s -- cycle) shifted point 6 of s;
picture P[];
P1 = image(
    for i=0 upto 3:
        fill t rotated 90i withcolor if odd i: Oranges 7 2 else: Blues 7 2 fi;
        draw t rotated 90i;
    endfor
);
P2 = image(
    t := t rotatedabout(point 0 of t, 180 - angle (point 2 of t - point 0 of t)) shifted (point 1 of t - point 0 of t);
    fill t withcolor Blues 7 2; draw t;
    t := t rotatedabout(point 1/2 of t, 180);
    fill t withcolor Blues 7 2; draw t;
    t := t rotatedabout(point 1 of t, -90);
    fill t withcolor Oranges 7 2; draw t;
    t := t rotatedabout(point 1/2 of t, 180);
    fill t withcolor Oranges 7 2; draw t;
    draw subpath (0, 2) of unitsquare scaled (abs(point 2 of t - point 1 of t) - abs(point 2 of t - point 0 of t)) shifted point 2 of t;
);

draw P1;
draw P2 shifted 180 right;
label.bot("\textit{Behold!}", point 1/2 of bbox currentpicture shifted 36 down);
\end{mplibcode}
$$

\vfill
\contrib{Bh\=askara (12th century)}

\section{The Pythagorean theorem III}

\vfill
$$
\begin{mplibcode}
path s, t, a, b, c;
s = fullcircle scaled 72;
t = (point 4 of s -- point 0 of s -- point sqrt(6) of s -- cycle) shifted point 6 of s;
a = unitsquare scaled abs(point 2 of t - point 0 of t) rotated angle (point 2 of t - point 0 of t) shifted point 0 of t;
b = unitsquare scaled abs(point 1 of t - point 2 of t) rotated angle (point 1 of t - point 2 of t) shifted point 2 of t;
c = unitsquare scaled abs(point 0 of t - point 1 of t) rotated angle (point 0 of t - point 1 of t) shifted point 1 of t;
color v, w; v = Oranges 7 1; w = Greens 7 1;
picture P[];
P0 = image(
    draw a;
    draw b;
    draw c;
);
P1 = image(
    fill a withcolor v;
    fill b withcolor v;
    draw P0
);
z0 = whatever[point 2 of a, point 3 of a] = whatever[point 2 of b, point 3 of b];
z1 = whatever[z0, point 3 of a]; x1 = xpart point 0 of a;
z2 = whatever[z0, point 2 of b]; x2 = xpart point 1 of b;
path wedge; wedge = subpath (0,1) of a -- subpath (0, 1) of b -- z2 -- z0 -- z1 -- cycle;

P2 = image(
    draw point 2 of a -- z0 -- point 3 of b dashed evenly scaled 1/2;
    path a', b'; numeric t, u;
    t = angle (point 1 of a - point 0 of a);
    u = angle (point 1 of b - point 0 of b);
    a' = a shifted - point 0 of a rotated -t slanted 1/2 rotated t shifted point 0 of a;
    b' = b shifted - point 0 of b rotated -u slanted -1/3 rotated u shifted point 0 of b;
    fill a' withcolor 1/4[v,w]; draw a';
    fill b' withcolor 1/4[v,w]; draw b';
    draw P0
);
P3 = image(
    draw point 2 of a -- z0 -- point 3 of b dashed evenly scaled 1/2;
    fill wedge withcolor 1/2[v,w]; draw wedge; draw point 1 of a -- z0;
    draw P0
);
P4 = image(
    fill wedge shifted (point 0 of a - z1) withcolor 3/4[v,w];
    draw wedge shifted (point 0 of a - z1);
    draw P0
);
P5 = image(
    fill c withcolor w;
    draw P0
);

draw P1;
draw P2 shifted (144,0);
draw P3 shifted (288,0);
draw P4 shifted (72, -200);
draw P5 shifted (216, -200);
\end{mplibcode}
$$
\vfill
\contrib{based on Euclid's proof}

\section{The Pythagorean theorem IV}

\vfill
$$
\begin{mplibcode}
path c, a, a', bq, bq'; numeric r; r = 59;
c = unitsquare shifted -(1/2, 1/2) scaled 160;
a = c scaled cosd(r) rotated r;
pair p, q;
p = whatever[point 0 of a, point 1 of a] = whatever[point 0 of c, point 1 of c];
q = whatever[point 0 of a, point 3 of a] = whatever[point 0 of c, point 3 of c];
bq = point 0 of c -- p -- point 0 of a -- q -- cycle;

fill a withcolor Blues 7 2;
for i=0 upto 3:
    fill bq rotated 90i withcolor if odd i: Oranges 7 2 else: Purples 7 2 fi;
    draw bq rotated 90i;
endfor

a' = a shifted (point 3 of c - point 0 of a);
fill a' withcolor Blues 7 2;
draw a';

bq' = bq rotated 180 shifted (point 1 of a' - point 2 of (bq rotated 180));
pair o; o = point 0 of bq';
for i=0 upto 3:
    fill bq' rotatedabout(o, 90i) withcolor if odd i: Oranges 7 2 else: Purples 7 2 fi;
    draw bq' rotatedabout(o, 90i);
endfor
\end{mplibcode}
$$
\vfill
\contrib{H.\@ E.\@ Dudeney (1917)}

\section{The Pythagorean theorem V}

\vfill
$$
\begin{mplibcode}
path t, t';
t = (origin -- 377 right -- 144 up -- cycle) scaled 3/4;
t' = t rotated -90 shifted (point 2 of t + point 1 of t rotated 90);

draw unitsquare scaled 8 withcolor 1/2;
draw unitsquare scaled 8 rotated -90 shifted point 0 of t' withcolor 1/2;
draw unitsquare scaled 8 rotated angle (point 1 of t - point 2 of t)
      shifted point 2 of t withcolor 1/2;

draw t;
draw t';
draw point 1 of t -- point 2 of t';

label.lft("$a$", point -1/2 of t);
label.bot("$b$", point 1/2 of t);
label.urt("$c$", point 3/2 of t);
label.top("$a$", point -1/2 of t');
label.lft("$b$", point 1/2 of t');
label.lrt("$c$", point 3/2 of t');

label.bot(btex \vbox{\openup 24pt\halign{\hfil $#$ \hfil\cr
A = 2 \cdot \frac12 ab + \frac12 c^2 = \frac12\left(a+b\right)^2\cr
c^2 = a^2 + b^2\cr}} etex, (xpart point 1 of t, ypart point 2 of t' - 12));
\end{mplibcode}
$$
\vfill
\contrib{James A.\@ Garfield (1876)}

\section{The Pythagorean theorem VI}

\vfill
$$
\begin{mplibcode}
numeric r;
r = 144;  z1 = r * dir 66;

draw unitsquare scaled 8 rotated 90 shifted (x1, 0) withcolor 1/2;
draw (left--right) scaled r withcolor Blues 7 7;
draw origin -- z1 -- (x1, 0) withcolor Blues 7 7;
draw fullcircle scaled 2r withcolor Reds 7 7;

label.top("$a$", (1/2 x1, 0));
label.rt("$b$", (x1, 1/2 y1));
label.ulft("$c$", 1/2 z1);
label.top("$c$", (-1/2 r, 0));
label.top("$c-a$", (1/2(r+x1), 0));

label.lft(btex \vbox{\openup 24pt\halign{\hfil $\displaystyle#$ \hfil\cr
{c+a\over b} = {b \over c-a} \cr
a^2 + b^2 = c^2\cr}} etex, point -1/2 of bbox currentpicture + 16 left);
\end{mplibcode}
$$
\vfill
\contrib{Michael Hardy}

\section{A Pythagorean theorem: $aa' = bb' + cc'$}

\vfill
$$
\begin{mplibcode}
path c; c = fullcircle scaled 377;
z0 = point 4 of c; z1 = point 0 of c; z2 = point 2.828 of c;
z3 = 5/16[z0, z1];
z4 = whatever [z1, z2];
z4 - z3 = whatever * (z2 - z0);
picture P;
P = image(
    draw unitsquare scaled 6 rotated angle (z0 - z2) shifted z2 withcolor 1/2;
    draw unitsquare scaled 6 rotated angle (z3 - z4) shifted z4 withcolor 1/2;
    draw z3 -- z1 -- z2 -- z0 -- z3 -- z4;

    label.bot ("$a$", 1/2[z0, z1] shifted 10 down); label.top("$a'$", 7/16[z3, z1]);
    label.ulft("$b$", 1/2[z0, z2]); label.ulft("$b'$", 1/2[z3, z4]);
    label.urt ("$c$", 1/2[z1, z2]); label.llft("$c'$", 9/16[z1, z4]);

);

draw P shifted 200 up;
x5 = x4; y5 = 0;
draw unitsquare scaled 6 shifted z5 withcolor Blues 7 4;
draw z4--z5 withcolor Blues 7 4;
draw P;
label.bot("$\scriptstyle x$", 1/2[z3, z5]) withcolor Blues 7 6;
label.bot("$\scriptstyle y$", 1/2[z1, z5]) withcolor Blues 7 6;

label.bot(btex \vbox{\openup 8pt\halign{\hfil $\displaystyle # $\hfil\cr
{x\over b'} = {b\over a} \implies {x\over b} = {b'\over a} \implies ax = bb';\cr
{y\over c'} = {c\over a} \implies {y\over c} = {c'\over a} \implies ay = cc';\cr
\therefore\quad aa' = a\left(x+y\right) = bb' + cc'.\cr
}} etex, point 1/2 of bbox currentpicture shifted 24 down);
\end{mplibcode}
$$
\vfill
\contrib{Enzo R.\@ Gentile}

\section{The rolling circle squares itself}

\vfill
$$
\begin{mplibcode}
numeric r; r = 64;
numeric pi; pi = 3.141592653589793;
path base, h, c, c', s;

base = (left--right) scaled 7/2r;
h = halfcircle rotated 180 scaled (pi * r + r);
c = fullcircle scaled 2r rotated 90 shifted point 0 of h shifted (0, r);
c' = fullcircle scaled 2r rotated 270 shifted point 4 of h shifted (-r, r);
s = unitsquare scaled (sqrt(pi) * r) rotated -90 shifted point 0 of c';

fill c withcolor Blues 7 1;
fill s withcolor Blues 7 1;

draw base withcolor 1/2;
draw subpath (0, 4 + 1/45 angle point 1 of s) of h withcolor 1/2;
draw subpath (4 + 1/45 angle point 1 of s, 4) of h withcolor Blues 7 3;
draw s;
draw point infinity of h -- point 2 of c' dashed evenly;

forsuffixes $=c, c':
    draw point 0 of $ -- center $ -- point 2 of $ dashed evenly;
    draw $; drawdot point 0 of $ withpen pencircle scaled dotlabeldiam;
endfor

drawarrow subpath (5/4, 1/4) of fullcircle scaled (2r + 16)
    shifted center c withcolor Reds 7 7;
\end{mplibcode}
$$
\vfill
\contrib{Thomas Elsner}

\section{On trisecting an angle}
\vfill
$$
\begin{mplibcode}
  picture link, pointer, pointer_groove;
  color metal, light_metal;
  metal = 1/256 (181, 166, 66);
  light_metal = 3/4[metal, white];

  link = image(
      path a, b, a', b', c;
      a = fullcircle scaled 3; a' = a shifted (98,0);
      b = fullcircle scaled 5; b' = b shifted center a';
      c = subpath(2,6) of b -- subpath(-2,2) of b' -- cycle;
      fill c withcolor light_metal; draw c;
      fill a withcolor metal; draw a;
      fill a' withcolor metal; draw a';
  );
  pointer = image(
      path a, b, c; numeric r;
      a = fullcircle scaled 18;
      b = fullcircle scaled 24;
      r = 1/3;
      c = subpath(r,8-r) of b --
          point 8-r of b shifted (10cm,0) --
          point 0   of b shifted (116mm,0) --
          point 8+r of b shifted (10cm,0) -- cycle;
      fill c withcolor light_metal;
      fill subpath (9,11) of c -- cycle withcolor 7/8 white;
      draw point 9 of c -- point 11 of c;
      draw c;
      fill a withcolor white; draw a;
  );
  pointer_groove = image(
      draw pointer;
      path g;
      g = (halfcircle scaled 4 rotated 90 --
           halfcircle scaled 4 rotated 270 shifted (4cm,0) -- 
           cycle) shifted (5cm,0);
      fill g withcolor 7/8[metal,white]; draw g;
  );
  draw pointer        rotated 42;
  draw pointer_groove rotated 28;
  draw pointer_groove rotated 14;
  draw pointer        rotated 0;
  z0 = 210 right rotated 14;
  z1 = 120 right;
  numeric t; t = angle (z0-z1);
  draw link rotated t shifted z1 rotatedabout(z0,-34.5);
  draw link rotated t shifted z1 rotatedabout(z0,-34.5) rotated 14;
  draw link rotated t shifted z1;
  draw link rotated t shifted z1 rotated 14;
\end{mplibcode}
$$
\vfill
\contrib{Rufus Isaacs}

\section{Trisection in an infinite number of steps}
\vfill
$$
\begin{mplibcode}
numeric alpha, beta;
alpha = 144;
beta = 0;
for i=1 upto 9:
    beta := beta if odd i: + else: - fi alpha * (2 ** -i);
    path ray;
    ray = origin -- (130 + 10i) * dir beta;
    draw ray withcolor 3/4;
    if i < 7:
        picture t;
        t = thelabel("$\frac1{" & decimal (2**i) & "}$", origin) scaled (1 - i/8) shifted point 1 of ray;
        unfill bbox t; draw t;
    fi
endfor

for i = 0 upto 3:
    draw origin -- 240 dir (i * alpha/3)
        if i mod 3 > 0: dashed evenly fi
        withcolor Reds 6 6;
endfor
filldraw fullcircle scaled dotlabeldiam;
label.bot("$\displaystyle \frac13 = \frac12 - \frac14 +\frac18 - \frac1{16} + \cdots $",
    point 1/2 of bbox currentpicture shifted 24 down);

\end{mplibcode}
$$
\vfill
\contrib{Eric Kincanon}

\section{Trisection of a line segment}
\vfill
$$
\begin{mplibcode}
picture P[];
pair A, B, C, D, E, F;
A = origin;
B = 80 right;
C = B rotated 60;
D = C rotated 180;
E = 1/2 [B, D];
F = p[A,B] = q[E, C];

path ca, cb;
ca = fullcircle scaled 2 abs (A-B);
cb = ca rotated 180 shifted B;

P0 = image(
    drawoptions(withpen pencircle scaled 1/4 withcolor Blues 7 4);
    draw subpath 1/45(50, 70) of ca; draw subpath 1/45(50, 70) of cb;
    draw subpath -1/45(50, 70) of ca; draw subpath -1/45(50, 70) of cb;
    drawoptions();
);
P9 = image(
    draw A -- B;
    dotlabel.ulft("$A$", A);
    dotlabel.urt("$B$", B);
);
P1 = image(draw P0; draw P9);
P2 = image(
    draw P0;
    drawoptions(withpen pencircle scaled 1/4 withcolor Blues 7 4);
    draw subpath 1/45(230, 250) of ca;
    drawoptions();
    draw C -- D withcolor 1/2;
    dotlabel.top("$C$", C);
    dotlabel.llft("$D$", D);
    draw P9;
);
P3 = image(
    draw B--D withcolor 1/2;
    draw A -- C reflectedabout(A,B) dashed evenly withcolor 1/2;
    drawdot E withpen pencircle scaled dotlabeldiam;
    label("$E$", E-(2,12));
    draw P2;
);
P4 = image(
    draw B--C dashed evenly withcolor 1/2;
    draw C--E withcolor 1/2;
    draw P3;
    dotlabel.lrt("$F$", F);
);

draw P1;
draw P2 shifted (250, 0);
draw P3 shifted (0, -220);
draw P4 shifted (250, -220);

label.bot("$\overline{AF} = \frac13\cdot\overline{AB}$",
    point 1/2 of bbox currentpicture shifted 36 down);
\end{mplibcode}
$$
\vfill
\contrib{Scott Cobel}

\section{The vertex angles of a star sum to 180°}
\vfill
$$
\begin{mplibcode}
z3 = -z5 = 120 left;
z1 = 180 dir 81;
z2 = 250 dir 130;
z4 =  90 dir -100;

z6 = z5 + 72 right;
z7 = whatever [z2, z5] = whatever [z1, z4];
z8 = whatever [z3, z5] = whatever [z1, z4];
y9 = y1; z9 - z5 = whatever * (z1 - z4);

path star;
star = z3 -- z5 -- z2 -- z4 -- z1 -- cycle;
draw star withcolor Blues 7 7;
draw z6 -- z5 -- z9 dashed evenly withcolor Blues 7 5;

def angle_point(expr a, b, c, r) = b + r * (unitvector(a-b) + unitvector(c-b)) enddef;

label("$α$", angle_point(z3, z1, z4, 12));
label("$β$", angle_point(z5, z2, z4, 16));
label("$γ$", angle_point(z5, z3, z1, 10));
label("$δ$", angle_point(z1, z4, z2, 12));
label("$ε$", angle_point(z2, z5, z3, 12));

label("$α+γ$", angle_point(z1, z8, z5, 16) + 8 down);
label("$α+γ$", angle_point(z9, z5, z6, 16) + 8 down);
label("$β+δ$", angle_point(z5, z7, z4, 18) + 1 up);
label("$β+δ$", angle_point(z2, z5, z9, 18) + 2 down);
\end{mplibcode}
$$
\vfill
\contrib{Fouad Nakhli}

\section{Viviani's theorem I}
\vfill
$$
\begin{mplibcode}
pair A', B', C', A, B, C, D, E, F, G, P, Q, R;
A' = origin;
B' = 300 right;
C' = B' rotated 60;
P = 3/8[B', C'];
xpart Q = xpart C';
ypart Q = ypart E = ypart P;
D = whatever[A', B']; xpart D = xpart P;
E = whatever[A', C'];
R = whatever[A', C']; R - P = whatever * (A' - C') rotated 90;
G = whatever[C', Q] = whatever [R, P];
A = whatever[A', B']; G-A = whatever * (C' - A');
B - B' = A - A' = C - C';
F = whatever[B, C]; F-P = whatever * (B-C) rotated 90;

def right_angle_mark(expr a, b, s) =
    subpath (1,3) of unitsquare scaled s rotated angle(b-a) shifted a
enddef;

drawoptions(withcolor 1/2);
draw right_angle_mark(D, B, 6);
draw right_angle_mark(F, P, 6);
draw right_angle_mark(G, P, 6);
draw right_angle_mark(Q, P, 6);
draw right_angle_mark(R, A', 6);
draw E--P;
draw A'--B'--C'--cycle;
drawoptions();

draw P--F withcolor Reds 7 7;
draw R--G--Q withcolor 1/2[Reds 7 7, white];
draw G--P dashed evenly scaled 3/4 withcolor Greens 7 7;
draw G--C' dashed evenly scaled 3/4 withcolor 1/2[Greens 7 7, white];

draw P--D dashed withdots scaled 1/4 withcolor Blues 7 7;
draw A--B--C--cycle;

drawdot P withpen pencircle scaled dotlabeldiam;

forsuffixes $=A, A', B, B', D, Q: label.bot("\strut$" & str $ & "$", $); endfor
forsuffixes $=C, C': label.top("$" & str $ & "$", $); endfor
label.urt("$F$", F);
label("$G$", G + 10 dir 192);

label.top(btex \vbox{\halign{\hss #\hss\cr
The perpendiculars to the sides from a point on\cr
the boundary or within an equilateral triangle\cr
add up to the height of the triangle.\cr
}} etex, point 5/2 of bbox currentpicture shifted 42 up);

label("\textit{This shows a particular example, with $C'GQ$ collinear, rather than the general case}",
point 1/2 of bbox currentpicture shifted 42 down);
\end{mplibcode}
$$
\vfill
\contrib{Samuel Wolf}

\section{Viviani's theorem II}
\vfill
$$
\begin{mplibcode}

def distance(expr a, b, c) = abs ypart ((a-b) rotated -angle (c-b)) enddef;

pair a, b, c, p;
a = 89 up;
b = a rotated 120;
c = b rotated 120;
p = 21 dir 42;

numeric h[];
h0 = distance(a, b, c);
h1 = distance(p, a, b);
h2 = distance(p, b, c);
h3 = distance(p, c, a);

path t[];
t0 = a--b--c--cycle;
t1 = t0 rotated -120 shifted -point 2 of t0 scaled (h1/h0) shifted p;
t2 = t0 shifted -point 0 of t0 scaled (h2/h0) shifted p;
t3 = t0 rotated +120 shifted -point 1 of t0 scaled (h3/h0) shifted p;

z0 = 1/3[1/2[point 2 of t1, point 1 of t3], point 0 of t0];
z1 = 2/3[point 0 of t1, point 3/2 of t1];

color s[];
s1 = Reds 7 2; s2 = Oranges 7 2; s3 = Blues 7 2;
picture p[];
forsuffixes $=1,2,3: p$ = image(fill t$ withcolor s$; draw t$ --point 3/2 of t$); endfor

picture P[];
P1 = image(draw p1; draw p2; draw p3; draw t0;);
P2 = image(
    path cor;
    cor = reverse fullcircle rotated 90 scaled 4/3 h1 scaled 15/16 shifted z1;
    drawarrow subpath 1/45(20, 100) of cor withcolor Reds 7 7;
    drawarrow subpath 1/45(140, 220) of cor withcolor Reds 7 7;
    drawarrow subpath 1/45(260, 340) of cor withcolor Reds 7 7;
    draw p1 rotatedabout(z1, -120); draw p2; draw p3; draw t0;);
P3 = image(
    path cor;
    cor = reverse fullcircle rotated 90 scaled 4/3 (h1+h3) scaled 7/8 shifted z0;
    drawarrow subpath 1/45(20, 100) of cor withcolor Reds 7 7;
    drawarrow subpath 1/45(140, 220) of cor withcolor Reds 7 7;
    drawarrow subpath 1/45(260, 340) of cor withcolor Reds 7 7;
    draw point 2 of t1 -- point 1 of t3 dashed withdots scaled 1/2;
    draw p2;
    draw p1 rotatedabout(z1, -120) rotatedabout(z0, -120);
    draw p3 rotatedabout(z0, -120);
    draw t0;);

draw P1 shifted 160 up;
draw P2 shifted 90 left;
draw P3 shifted 90 right;
label.top(btex \vbox{\halign{\hss #\hss\cr
The perpendiculars to the sides from a point on\cr
the boundary or within an equilateral triangle\cr
add up to the height of the triangle.\cr
}} etex, point 5/2 of bbox currentpicture shifted 42 up);
\end{mplibcode}
$$
\vfill
\contrib{Ken-Ichiroh Kawasaki}

\section{A theorem about right angles}
\vfill
$$
\begin{mplibcode}

path s, t;
s = unitsquare shifted -(1/2,1/2) scaled 210;
t = subpath (3, 2) of s -- point 1.732 of fullcircle scaled 210
  shifted point 5/2 of s -- cycle;

fill t withcolor Blues 7 1;
fill s withcolor Oranges 7 1;
for i=0 upto 3: draw t rotated 90i; endfor;
draw point 2 of t -- point 2 of t rotated 180;

label.top(btex \vbox{\halign{\hss #\hss\cr
The internal bisector of the right angle of a right\cr
triangle bisects the square on the hypotenuse\cr
}} etex, point 5/2 of bbox currentpicture shifted 42 up);
\end{mplibcode}
$$
\vfill
\contrib{Roland H.\@ Eddy}

\section{Area and the projection theorem of a right triangle}
\vfill
$$
\begin{mplibcode}
def angle_arc(expr a, o, b, r) = fullcircle scaled 2r rotated angle (a-o) shifted o cutafter (o--b) enddef;

path c; c = fullcircle scaled 180;
pair A, B, C, D, E, F;
A = point 4 of c;
B = point 0 of c;
C = point 1.7 of c;
D = (xpart C, ypart A);
E = C rotatedabout(D, -90);
F = B rotatedabout(D, -90);

color r, b, g; r = Reds 7 1; g = Greens 7 1; b = Blues 7 1;
picture P[];
P1 = image(
    fill A--B--C--cycle withcolor r;
    draw unitsquare scaled 6 rotated angle (A-C) shifted C withcolor 3/4 r;
    draw unitsquare scaled 6 rotated angle (C-D) shifted D withcolor 3/4 r;
    draw angle_arc(D, A, C, 12) withpen pencircle scaled 1 withcolor Reds 7 5;
    draw angle_arc(D, C, B, 12) withpen pencircle scaled 1 withcolor Reds 7 5;
    draw D--C--B--A--C;
    label.bot("$A$", A);
    label.top("$C$", C);
    label.bot("$D$", D);
    label.bot("$B$", B);
);

P2 = image(
    fill A--D--C--cycle withcolor r;
    fill A--D--F--cycle withcolor g;
    fill F--D--E--cycle withcolor r;
    fill C--D--E--cycle withcolor b;
    draw unitsquare scaled 6 rotated angle (C-D) shifted D withcolor 3/4 r;
    draw angle_arc(D, A, C, 12) withpen pencircle scaled 1 withcolor Reds 7 5;
    draw angle_arc(D, E, F, 12) withpen pencircle scaled 1 withcolor Reds 7 5;
    drawarrow subpath (7/4, 1/4) of quartercircle scaled 42 shifted D withcolor Blues 6 6;
    draw A--F--E--C--A--E;
    draw C--F;
    label.lft ("$A$", A);
    label.top ("$C$", C);
    label.llft("$D$", D);
    label.rt  ("$E$", E);
    label.bot ("$F$", F);
);

P3 = image(
    fill A--D--C--cycle withcolor r;
    fill C--D--E--cycle withcolor b;
    z3 = whatever[A,C]; z3 - E = whatever * (A-C) rotated 90;
    begingroup; interim ahangle := 180;
    drawarrow E--z3 dashed evenly scaled 3/4 withpen pencircle scaled 1/4;
    label.urt("$h$", 1/4[z3, E]);
    endgroup;
    draw A--E--C--A;
    draw C--D;
    label.bot("$A$", A);
    label.top("$C$", C);
    label.bot("$D$", D);
    label.bot("$E$", E);
);

P4 = image(
    fill A--D--C--cycle withcolor r;
    fill A--D--F--cycle withcolor g;
    z4 = whatever[A,C]; z4 - F = whatever * (A-C) rotated 90;
    draw z4--F dashed evenly scaled 3/4 withpen pencircle scaled 1/4;
    label.urt("$h$", 1/4[z4, F]);
    draw A--C--F--A--D;
    label.lft ("$A$", A);
    label.urt ("$C$", C);
    label.rt  ("$D$", D);
    label.lrt ("$F$", F);
);

P5 = image(
    fill C--D--E--cycle withcolor b;
    draw D--E--C--D;
    label.top("$C$", C);
    label.bot("$D$", D);
    label.bot("$E$", E);
);

P6 = image(
    fill A--D--F--cycle withcolor g;
    draw A--F--D--A;
    label.lft ("$A$", A);
    label.rt  ("$D$", D);
    label.lrt ("$F$", F);
);

draw P1 shifted 120 left;
draw P2 shifted 120 right;
numeric y; y = -190;
draw P3 shifted (-120, y); label("${}={}$", (12, y+16)); draw P4 shifted (+120, y);
y := y - 112;
draw P5 shifted (-120, y-abs(D-B)); label("${}={}$", (12, y-28)); draw P6 shifted (+120, y);
label("$CD^2 = AD\cdot DB$", point 1/2 of bbox currentpicture shifted 42 down);
\end{mplibcode}
$$
\vfill
\contrib{Sidney H.\@ Kung}

\section{Chords and tangents of equal length}
\vfill
$$
\begin{mplibcode}
def angle_arc(expr a, o, b, r) = fullcircle scaled 2r rotated angle (a-o) shifted o cutafter (o--b) enddef;

path C[]; pair O, P, Q, R;
C1 = fullcircle scaled 280; O = point 0 of C1;
C2 = fullcircle scaled 200 shifted O;

numeric t, u;
(t, u) = C1 intersectiontimes C2;
P = point t of C1;
Q = point 8-t of C1;
z0 = whatever[P, P + direction t of C1]; y0 = ypart point 6 of C1;
R = C2 intersectionpoint (z0--P);

draw center C1 -- P -- point 4 of C1 -- O withcolor 7/8;

forsuffixes $=P, Q, R: 
    draw O -- $ withcolor Blues 7 6; 
endfor
draw 5/4[Q, P] -- 5/4[P, Q] withcolor Reds 7 6;
draw 5/4[P, R] -- 5/4[R, P] withcolor Reds 7 6;

draw angle_arc(O, Q, P, 30);
draw angle_arc(O, Q, P, 28);
draw angle_arc(O, P, R, 30);
draw angle_arc(O, P, R, 28);

draw C1 withcolor 1/2;
draw C2 withcolor 1/2;;

dotlabel.urt("$O$", O);
dotlabel.urt("\strut $P$", P);
dotlabel.lrt("\strut $Q$", Q);
dotlabel.rt("$\;R$", R);

label.top(btex \vbox{\openup6pt\halign{\hss #\hss\cr
If circle $C_1$ passes through the center $O$ of circle $C_2$, the length\cr
of the common chord $\overline{PQ}$ is equal to the tangent segment $\overline{PR}$.\cr
}} etex, point 5/2 of bbox currentpicture shifted 42 up);
\end{mplibcode}
$$
\vfill
\contrib{Roland H.\@ Eddy}

\section{Completing the square}
\vfill
$$
\begin{mplibcode}
path xx, ax, hax, haha;
numeric x, a;
x = 89; a = 34;
xx   = unitsquare shifted 1/2 left scaled x shifted 12 up;
ax   = unitsquare shifted 1/2 left xscaled x yscaled -a shifted 12 down;
hax  = unitsquare shifted 1/2 left xscaled x yscaled -1/2 a shifted 12 down;
haha = unitsquare scaled 1/2 a rotated -90 shifted point 1 of xx shifted (8, -8);

picture P[];
P1 = image(
    fill xx withcolor Oranges 7 1; draw xx; 
    label.top("$x$", point 5/2 of xx); 
    label.lft("$x$", point 7/2 of xx); 
    label("${}+{}$", origin);
    fill ax withcolor Blues 7 2; draw ax;
    label.lft("$a$", point 7/2 of ax);
);

P2 = image(
    fill xx withcolor Oranges 7 1; draw xx; 
    label("${}+{}$", origin);
    for i=0, 1:
        fill hax shifted (0, -24i) withcolor Blues 7 2; 
        draw hax shifted (0, -24i);
    endfor
);

P3 = image(
    fill xx withcolor Oranges 7 1; draw xx; 
    hax := hax shifted (point 0 of xx - point 0 of hax);
    fill hax withcolor Blues 7 2; draw hax;
    hax := hax shifted - point 0 of hax rotated 90 shifted point 1 of xx;
    fill hax withcolor Blues 7 2; draw hax;

    fill haha withcolor Blues 7 1;
    draw haha dashed withdots scaled 1/4;
);


draw P1 shifted 144 left;
label("$=$", (-72, 16));
draw P2;
label("$=$", (72, 16));
draw P3 shifted 144 right;

label.top("$x^2 + ax = \left(x + a/2\right)^2 - \left(a/2\right)^2$", 
point 5/2 of bbox currentpicture shifted 42 up);
\end{mplibcode}
$$
\vfill
\contrib{Charles D.\@ Gallant}

\section{Algebraic areas I}
\vfill
$$
\begin{mplibcode}
numeric a, b;
a = 89; b = 21;
picture P[];
P1 = image(
    fill unitsquare xscaled a yscaled b shifted (0, a) withcolor Greens 7 1;
    fill unitsquare xscaled b yscaled a shifted (a, 0) withcolor Greens 7 1;
    draw (a, 0) -- (a, a+b) dashed withdots scaled 1/4;
    draw (0, a) -- (a+b, a) dashed withdots scaled 1/4;
    draw (a-b, a) -- (a-b, a+b) dashed withdots scaled 1/4;
    draw (a, a-b) -- (a+b, a-b) dashed withdots scaled 1/4;
    draw unitsquare scaled (a+b);
    label.bot("\strut $a$", (1/2a, 0));
    label.bot("\strut $b$", (a+1/2b, 0));
    label.lft("$a$", (0, 1/2a));
    label.lft("$b$", (0, a+1/2b));
);
P2 = image(
    draw unitsquare scaled (a-b);
    label.bot("\strut $a-b$", 1/2(a-b, 0));
);
P3 = image(
    draw unitsquare scaled a;
    draw unitsquare scaled b shifted (a,a);
    label.bot("\strut $a$", (1/2a, 0));
    label("$b$", (a + 1/2b, a + 1/2b));
);
P4 = image(
    fill unitsquare scaled a withcolor Greens 7 1;
    fill unitsquare scaled b shifted (a,a) withcolor Greens 7 1;
    fill unitsquare scaled (a-b) withcolor background;
    draw (a-b, a) -- (a-b, a-b) -- (a, a-b) dashed withdots scaled 1/4;
    draw (0, a-b) -- (a-b, a-b) -- (a-b, 0);
    draw unitsquare scaled a;
    draw unitsquare scaled b shifted (a,a);
    label.bot("\strut $a-b$", 1/2(a-b, 0));
    label.bot("\strut $b$", (a-1/2b, 0));
    label("$b$", (a + 1/2b, a + 1/2b));
);


numeric x, y;
draw P1;

y = 3/4 (a-b);
x = a + b + 14;
label("$+$", (x, y));

x := x + 14;
draw P2 shifted (x, 0);

x := x + a - b + 16;
label("$=$", (x, y));

x := x + 16;
draw P3 shifted (x, 0);

x := x + a + 14;
label("$+$", (x, y));

x := x + 14;
draw P4 shifted (x, 0);

label.top("$\left(a+b\right)^2 + \left(a-b\right)^2 = 2\left(a^2 + b^2\right)$", 
point 5/2 of bbox currentpicture shifted 42 up);
\end{mplibcode}
$$
\vfill
\contrib{Shirley Wakin}

\section{Algebraic areas II}
\vfill
$$
\begin{mplibcode}
input arrow_label

numeric a, b, c;
a = 80; 2b = a; 2c = b;

def make_box(expr p, shade) = image(fill p withcolor shade; draw p) enddef;


path s[];
s1 = unitsquare scaled (a-b-c); 
s2 = unitsquare scaled (2c); 
s3 = unitsquare scaled (a-b+c);
s4 = unitsquare scaled (2b);
s5 = unitsquare scaled (a+b-c); 
s6 = unitsquare xscaled (a+b-c) yscaled (a-b+c);
s7 = unitsquare xscaled (a-b+c) yscaled (a+b-c);

picture t[];
t1 = make_box(s1, Reds 7 2);
t2 = make_box(s2, Oranges 7 2);
t3 = make_box(s3, YlGn 7 2);
t4 = make_box(s4, Greens 7 2);
t5 = make_box(s5, Blues 7 2);
t6 = make_box(s6, Purples 7 2);
t7 = make_box(s7, Purples 7 2);

picture P[];
P1 = image(
    draw t5;
    draw t7 shifted point 1 of s5;
    draw t1 shifted (point 2 of s5 - point 3 of s1);
    draw t3 shifted point 2 of s5;
    draw t6 shifted point 3 of s5;

    draw t4 shifted (2a + 30, 0);
    draw t2 shifted (2a + 30 + 2b + 30, 0);

    arrow_label(origin, 2a * right, "$2a$", 9);
    label.bot("\strut$2b$", (2a + 30 + b, 0));
    label.bot("\strut$2c$", (2a + 30 + 2b + 30 + c, 0));
);
P2 = image(
    draw t4;
    draw t7 shifted point 1 of s4;
    draw t6 shifted point 3 of s4;
    draw t2 shifted ((1,1) scaled (a+b-c));

    draw t5 shifted (a + b + c + 20, 0);
    draw t3 shifted (2a + 2b + 40, 0);
    draw t1 shifted (3a + b + c + 60, 0);

    arrow_label(origin, 2b * up, "$2b$", -12);
    arrow_label((0, a+b+c), (a+b+c, a+b+c), "\strut$a+b+c$", -12);
    label.rt("$2c$", (a+b+c, a+b));
    label.top("$a+b-c$", (3/2a + 3/2b + 1/2c + 20, a + b - c + 4));
    label.top("$a-b+c$", (5/2a + 3/2b + 1/2c + 40, a - b + c + 4));
    label.top("$a-b-c$", (7/2a + 1/2b + 1/2c + 60, a - b - c + 4));
);

label.top(P1, origin);
label.top(P2, (0, 2a+2b));


label.top(btex $\left(a+b+c\right)^2 
          + \left(a+b-c\right)^2 
          + \left(a-b+c\right)^2 
          + \left(a-b-c\right)^2 
          = \left(2a\right)^2 
          + \left(2b\right)^2
          + \left(2c\right)^2$ etex, 
point 5/2 of bbox currentpicture shifted 42 up);
\end{mplibcode}
$$
\vfill
\contrib{Sam Pooley and K.\@ Ann Drude}

\section{Sum of squares identity}
\vfill
$$
\begin{mplibcode}
input arrow_label
picture P[];
numeric a,b,c,d;
a = sqrt(90); 1.732a = 1.414b; c + d = 3/2a; 1.414c = d;

P1 = image(
    path s[];
    s1 = unitsquare xscaled -(a*a) yscaled -(d*d);
    s2 = unitsquare xscaled  (b*b) yscaled -(d*d);
    s3 = unitsquare xscaled  (b*b) yscaled (c*c);
    s4 = unitsquare xscaled -(a*a) yscaled (c*c);
                             
    fill s1 withcolor Reds 7 1;     draw s1;
    fill s2 withcolor Greens 7 1;   draw s2;
    fill s3 withcolor Oranges 7 1;  draw s3;
    fill s4 withcolor Blues 7 1;    draw s4;

    label("$a^2d^2$", center s1);  
    label("$b^2d^2$", center s2);  
    label("$b^2c^2$", center s3);  
    label("$a^2c^2$", center s4);  

    arrow_label(point 3 of s4, point 2 of s4, "$a^2$", 8);    
    arrow_label(point 2 of s4, point 1 of s4, "$c^2$", 8);    
    arrow_label(point 2 of s3, point 3 of s3, "$b^2$", 8);    
    arrow_label(point 1 of s1, point 2 of s1, "$d^2$", 8);    
    
);
P2 = image(
    path s[];
    s1 = unitsquare xscaled -(a*d) yscaled -(a*d);
    s2 = unitsquare xscaled  (b*d) yscaled -(b*d);
    s3 = unitsquare xscaled  (b*c) yscaled (b*c);
    s4 = unitsquare xscaled -(a*c) yscaled (a*c);
                             
    fill s1 withcolor Reds 7 1;     draw s1;
    fill s2 withcolor Greens 7 1;   draw s2;
    fill s3 withcolor Oranges 7 1;  draw s3;
    fill s4 withcolor Blues 7 1;    draw s4;

    label("$\left(ad\right)^2$", center s1);  
    label("$\left(bd\right)^2$", center s2);  
    label("$\left(bc\right)^2$", center s3);  
    label("$\left(ac\right)^2$", center s4);  
) shifted (260, 0);
P3 = image(
    path s[];
    s1 = unitsquare xscaled -(a*d) yscaled -(a*d);
    s2 = unitsquare xscaled  (b*d) yscaled -(b*d) shifted (44, -12);
    s3 = unitsquare xscaled  (b*c) yscaled (b*c);
    s4 = unitsquare xscaled -(a*c) yscaled (a*c) shifted point 1 of s2;
                             
    fill s1 withcolor Reds 7 1;     draw s1;
    fill s2 withcolor Greens 7 1;   draw s2;
    fill s3 withcolor Oranges 7 1;  draw s3;
    fill s4 withcolor Blues 7 1;    draw s4;

    label("$\left(ad\right)^2$", center s1);  
    label("$\left(bd\right)^2$", center s2);  
    label("$\left(bc\right)^2$", center s3);  
    label("$\left(ac\right)^2$", center s4);  
    
    pair t, u, v; 
    t = point arctime 2b*d-a*c of s2 of s2;
    u = whatever[point 3 of s2, point 4 of s2]; ypart u = ypart t;
    v = whatever[point 1 of s4, point 2 of s4]; ypart v = ypart t;
    draw t--u dashed withdots scaled 1/2;
    draw v--point 1 of s4 dashed withdots scaled 1/2;

    arrow_label(point 2 of s2, t, "$ac$", 8);
    arrow_label(t, point 3 of s4, "$bd$", 8);
) shifted (-40, -170);
P4 = image(
    path s[];
    s1 = unitsquare xscaled -(a*d) yscaled -(a*d);
    s3 = unitsquare xscaled  (b*c) yscaled (b*c);

    s21 = unitsquare xscaled -(a*c) yscaled (b*d) shifted t;
    s22 = unitsquare xscaled -(b*d) yscaled (-a*c) shifted (t + 7 down);
    s4 = unitsquare scaled (b*d-a*c) shifted (u + 7 left);
                             
    fill s1 withcolor Reds 7 1;     draw s1;
    fill s21 withcolor 1/2[Blues 7 1, Greens 7 1];   
    fill s22 withcolor 1/2[Blues 7 1, Greens 7 1];   
    draw s21; draw s22;
    fill s3 withcolor Oranges 7 1;  draw s3;
    fill s4 withcolor Purples 7 2;    draw s4;

    label("$\left(ad\right)^2$", center s1);  
    label("$abcd$", center s21);  
    label("$abcd$", center s22);  
    label("$\left(bc\right)^2$", center s3);  
    label("$\scriptstyle \left(bd-ac\right)^2$", center s4);  

    label.top("$ac$", point 5/2 of s21);
    label.rt("$bd$", point 7/2 of s21);
    label.rt("$ac$", point 7/2 of s22);
    label.bot("$bd$", point 5/2 of s22);
    
) shifted (260, -180);
P5 = image(
    path s[];
    s1 = unitsquare xscaled -(a*d) yscaled -(a*d);
    s3 = unitsquare xscaled  (b*c) yscaled (b*c);

    s21 = unitsquare xscaled -(a*d) yscaled (b*c) shifted (-12, 12);
    s22 = unitsquare xscaled (b*c) yscaled -(a*d) shifted (12, -12);
    s4 = unitsquare scaled (b*d-a*c) shifted point 1 of s22 shifted 12 right;
                             
    fill s1 withcolor Reds 7 1;     draw s1;
    fill s21 withcolor 1/2[Blues 7 1, Greens 7 1];   
    fill s22 withcolor 1/2[Blues 7 1, Greens 7 1];   
    draw s21; draw s22;
    fill s3 withcolor Oranges 7 1;  draw s3;
    fill s4 withcolor Purples 7 2;  draw s4;

    label("$\left(ad\right)^2$", center s1);  
    label("$abcd$", center s21);  
    label("$abcd$", center s22);  
    label("$\left(bc\right)^2$", center s3);  
    label("$\scriptstyle \left(bd-ac\right)^2$", center s4);  

    label.top("$ad$", point 5/2 of s21);
    label.lft("$bc$", point 3/2 of s21);
    label.rt("$ad$", point 3/2 of s22);
    label.bot("$bc$", point 5/2 of s22);
    
) shifted (-40, -380);
P6 = image(
    path s[];
    s1 = unitsquare scaled (a*d + b*c);
    s4 = unitsquare scaled (b*d-a*c) shifted point 1 of s1 shifted 12 right;
                             
    fill s1 withcolor 1/2[1/2[Reds 7 1, Oranges 7 1], 1/2[Blues 7 1, Greens 7 1]];   
    draw s1;
    fill s4 withcolor Purples 7 2;  draw s4;

    label("$\left(ad + bc\right)^2$", center s1);  
    label("$\scriptstyle \left(bd-ac\right)^2$", center s4);  
) shifted (180, -480);


draw P1;
draw P2;
draw P3;
draw P4;
draw P5;
draw P6;

def connect_with_arrow(expr a, b) = 
    drawarrow (left-- 4 right) scaled 4 rotated angle (b-a) shifted 1/2[a,b] withcolor Blues 5 5;
enddef;
connect_with_arrow(center P1 + 10 right, center P2);
connect_with_arrow(center P2, center P3);
connect_with_arrow(center P3, center P4);
connect_with_arrow(center P4, center P5);
connect_with_arrow(center P5, center P6);

label.top(btex $\left(a^2+b^2\right)\left(c^2+d^2\right)
          = \left(ab + bc\right)^2 
          + \left(bd-ac\right)^2 $ etex, 
point 5/2 of bbox currentpicture shifted 42 up);
\end{mplibcode}
$$
\vfill
\contrib{Diophantus of Alexandria}

\section{Polygonal numbers}

\vfill
$$
\begin{mplibcode}
vardef around(expr p, r) = 
    if pair p:
        fullcircle scaled 2r shifted p
    elseif path p and (length(p) = 0):
        fullcircle scaled 2r shifted point 0 of p
    elseif path p:
        for i=1 upto length(p):
            subpath (i-1, i) of p 
            shifted (r * unitvector(direction i-1/2 of p rotated -90))
            ..
        endfor 
        if not cycle p:
            for i=length(p) downto 1:
                subpath (i, i-1) of p
                shifted (r * unitvector(direction i-1/2 of p rotated 90))
                ..
            endfor
        fi cycle
    fi
enddef;
% k-th n-gonal number...
numeric k, n;  
k = 6;
n = 6;
path gon[];

for i=2 upto k:
    gon[i] = (origin for j=1 upto n-1: -- dir (180/n*j) endfor -- cycle) scaled 50(i-1);
endfor

numeric r; r = 8;

fill around(origin, r) withcolor Blues 7 3; 
draw around(origin, r); 

for i=1 upto n-1:
    path a; a = around(point i of gon2 -- point i of gon[k], r);
    fill a withcolor Oranges 7 1; draw a;
endfor

for i=2 upto n-2:
    draw origin -- point i of gon[k] dashed evenly;
endfor

for i=1 upto n-2:
    path a; a = around(
        point i+1/2 of gon[3] -- 
        point i+1/(k-1) of gon[k] -- 
        point i+1-1/(k-1) of gon[k] -- cycle, r);
    fill a withcolor Greens 7 1; draw a;
endfor

for i=2 upto k:
    draw gon[i];
    for j = i-1 upto (n-1)*i:
        drawdot point j/(i-1) of gon[i] withpen pencircle scaled r;
    endfor
endfor
drawdot origin withpen pencircle scaled r;

label.top(btex The $k$\textsuperscript{th} $n$-gonal number is 
$1 + \bigl(k-1\bigr)\bigl(n-1\bigr) + \frac12\bigl(k-2\bigr)\bigl(k-1\bigr)\bigl(n-2\bigr)$ etex, 
point 5/2 of bbox currentpicture shifted 42 up);
\end{mplibcode}
$$
\vfill
\contrib{Dave Logothetti}

\section{The volume of a frustrum of a square pyramid}

\vfill
$$
\begin{mplibcode}
input isometric-projection
set_projection(22, -34);

path base, hlid, mlid;
numeric h, a, b; h = 6; b = 7; a = 3;
base = p(0, 0, 0) -- p(0, 0, b) -- p(-b, 0, b) -- p(-b, 0, 0) -- cycle;
hlid  = p(0, h, 0) -- p(0, h, a) -- p(-a, h, a) -- p(-a, h, 0) -- cycle;
mlid  = p(0, b-a, 0) -- p(0, b-a, a) -- p(-a, b-a, a) -- p(-a, b-a, 0) -- cycle;

picture P[];
P1 = image(
    path lid; lid = hlid;
    fill subpath (0, 1) of base -- subpath (1, 0) of lid -- cycle withcolor Blues 8 1;
    fill lid withcolor Blues 8 2;
    drawoptions(dashed withdots scaled 1/2 withcolor 1/2);
    draw subpath (1, 3) of base;
    draw point 2 of base -- point 2 of lid;

    drawoptions(withcolor 1/2);
    numeric t; t = 1/2;
    draw p(-t, 0, 0) -- p(-t, t, 0) -- p(0, t, 0) -- p(0, t, t) -- p(0, 0, t);
    draw p(-t, h, 0) -- p(-t, h-t, 0) -- p(0, h-t, 0) -- p(0, h-t, t) -- p(0, h, t);

    drawoptions();
    draw lid -- point 0 of base;
    draw point 3 of lid -- subpath (-1, 1) of base -- point 1 of lid;

    label.lft("$h$", p(0, 1/2 h, 0));
    label.urt("$a$", point 7/2 of lid);
    label.ulft("$a$", point 1/2 of lid);
    label.lrt("$b$", point 1/2 of base);
    label.llft("$b$", point 7/2 of base);

    label("$P_1$", p(0, -1, 0));
);
P2 = image(
    path lid; lid = mlid;
    fill subpath (0, 1) of base -- subpath (1, 0) of lid -- cycle withcolor Blues 8 1;
    fill lid withcolor Blues 8 2;
    drawoptions(dashed withdots scaled 1/2 withcolor 1/2);
    draw subpath (1, 3) of base;
    draw point 2 of base -- point 2 of lid;

    drawoptions(withcolor 1/2);
    numeric t; t = 1/2;
    draw p(-t, 0, 0) -- p(-t, t, 0) -- p(0, t, 0) -- p(0, t, t) -- p(0, 0, t);
    draw p(-t, b-a, 0) -- p(-t, b-a-t, 0) -- p(0, b-a-t, 0) -- p(0, b-a-t, t) -- p(0, b-a, t);

    drawoptions();
    draw lid -- point 0 of base;
    draw point 3 of lid -- subpath (-1, 1) of base -- point 1 of lid;

    label.lft("$b-a$", p(0, 1/2 (b-a), 0));
    label.urt("$a$", point 7/2 of lid);
    label.ulft("$a$", point 1/2 of lid);
    label.lrt("$b$", point 1/2 of base);
    label.llft("$b$", point 7/2 of base);

    label("$P_2$", p(0, -1, 0));
);
P3 = image(
    path aaa, bbb;
    aaa = (p(0, 0, 0) -- 
           p(0, 0, a) --
           p(0, a, a) --
           p(-a, a, a) --
           p(-a, a, 0) -- 
           p(-a, 0, 0) -- cycle) shifted p(0, b-a, 0);
    bbb = p(0, 0, 0) -- 
          p(0, 0, b) --
          p(0, b, b) --
          p(-b, b, b) --
          p(-b, b, 0) -- 
          p(-b, 0, 0) -- cycle;
    pair c; c = p(-a, b-a, a);
    
    drawoptions(withcolor Blues 8 1);
    fill subpath(0, 2) of bbb -- subpath (2, 0) of aaa -- cycle;
    fill subpath (3, 5) of aaa -- c -- cycle;

    drawoptions(withcolor Blues 8 2);
    fill subpath(2, 4) of bbb -- subpath (4, 2) of aaa -- cycle;
    fill subpath (-1, 1) of aaa -- c -- cycle;

    drawoptions(dashed withdots scaled 1/2 withcolor 1/2);
    draw subpath (1, 3) of base;
    draw p(-b, b, b) -- point 2 of base -- point 2 of lid;

    drawoptions(withcolor 1/2);
    for $=1,3,5: draw point $ of aaa -- c; endfor

     drawoptions();
    for $=1 upto length aaa: draw point $ of aaa -- point $ of bbb; endfor
    draw aaa;
    draw bbb;

    label.urt ("$a$", point 7/2 of lid);
    label.ulft("$a$", point 1/2 of lid);
    label.lft ("$a$", point 3/2 of aaa);
    
    label.rt("$b$", point 3/2 of bbb);
    label.lrt("$b$", point 1/2 of base);
    label.llft("$b$", point 7/2 of base);


    label("$3P_2$", p(0, -1, 0));
);

draw P1 shifted 120 left;
draw P2 shifted 120 right;
draw P3 shifted 240 down;

label.top(btex  $\displaystyle
    V_{P_1} = {h\over b-a}\cdot V_{P_2} = {h\over b-a}\cdot{1\over3}\left(b^3-a^3\right)
     = {h\over3}\left(a^2+ab+b^2\right)$ etex, 
     point 1/2 of bbox currentpicture shifted 42 down);
\end{mplibcode}
$$
\vfill
\contrib{\textit{The Moscow Papyrus}, c.\@ 1850 BCE}

\section{The volume of a hemisphere via Cavalieri's Principle}

\vfill
$$
\begin{mplibcode}
input arrow_label
input isometric-projection
set_projection(18, -32);

numeric r, h, s, tau;
tau = 6.283185307179586;
r * tau = 400 / ipscale;  
h = 3/4 r;
s = r +-+ h;

z0 = p(0,0,0); 
z1 = p(0,0,r);
z2 = p(0,r,r);
z3 = p(0,r,0);
z4 = p(tau * r, 0, 0);

z5 = p(tau * (r-h), h, 0);
z6 = p(tau * (r-h), h, h);
z7 = p(0, h, r);
z8 = p(0, h, 0);

 z9 = p(0, 0, 5r);
z10 = p(0, r, 5r);
z11 = p(0, h, 5r);

z12 = z9 shifted p(2r, 0, 0);
z13 = z10 shifted p(r, 0, 0);
z14 = 1/2[z9, z12];
z15 = z14 shifted p(0, 0, -r);
z16 = z14 shifted p(0, 0, +r);

z17 = z14 + p(-s, h, 0);
z18 = z14 + p(0, h, -s);
z19 = z14 + p(+s, h, 0);
z20 = z14 + p(0, h, +s);

path disc, base, arc, arch; 
base = for i=0 upto 11: z14 + p(r*cosd(30i), 0, r*sind(30i)) .. endfor cycle;
disc = for i=0 upto 11: z14 + p(s*cosd(30i), h, s*sind(30i)) .. endfor cycle;
numeric a, b;
a = directiontime down of base;
b = directiontime up of base;

arc = point a of base .. point a of disc .. z13 .. point b of disc .. point b of base;

drawoptions(dashed evenly withpen pencircle scaled 1/4 withcolor 1/2);
draw z1--z9--z12; draw z2--z10--z13; draw z7--z11;
draw z9--z10;

drawoptions(dashed withdots scaled 1/4 withcolor 1/2);
draw z0--z1--z4; draw z1--z2; 
draw z14 -- center disc;
fill disc withcolor Blues 7 1;
draw z13 -- center disc -- z19 -- z14;
draw subpath (b, a) of base;
draw subpath (b, a) of disc;

fill z5--z6--z7--z8--cycle withcolor Oranges 7 1;
draw z6--z7--z8;

drawoptions(dashed evenly withpen pencircle scaled 1/4 withcolor 1/2);
draw z11 -- center disc;

drawoptions();
draw z0--z4--z2--z3--z0; draw z3--z4; draw z8--z5--z6;
draw arc;
draw subpath (a, 12 + b) of base;
draw subpath (a, 12 + b) of disc;

path circ, trap, crad;
circ = fullcircle scaled (2s*ipscale) shifted center disc shifted 144 up;
crad = center circ -- point 3/4 of circ;
trap = (origin -- (tau*(r-h), 0) -- (tau*(r-h), h) -- (0, r) -- cycle) 
     shifted (0, -1/2h)
     scaled ipscale shifted (0, ypart center circ);
fill circ withcolor Blues 7 1;  
draw circ;
fill trap withcolor Oranges 7 1; draw trap;

label("${}={}$", 1/2[point 3/2 of trap, point 4 of circ]); 

drawoptions(withcolor Blues 7 6);
arrow_label(3/4[z1, z9], 3/4[z7, z11], "$h$", 0);
arrow_label(z0 shifted p(0,0,-3/4), z3 shifted p(0,0,-3/4), "$r$", 0);
arrow_label(z0 shifted p(0,0,-3/2), z4 shifted p(0,0,-3/2), "$2\pi r$", 0);
arrow_label(point 0 of trap, point 1 of trap, "$2\pi(r-h)$", 9);
label.lft("$r$", point -1/2 of trap);
label.rt("$h$", point 3/2 of trap);
label.lrt("$h$", 5/8[z5, z6]);
label.top("$r$", 1/2[z2, z3]);
label.top("$r$", 1/2[z0, z1]);
label.lrt("$r$", 1/2[z14, z19]);
drawdblarrow crad;
label.ulft("$\sqrt{r^2-h^2}$", point 5/8 of crad);

drawoptions();
label.urt("$S$", point 7/2 of arc);
label.urt("$P$", 5/8[z2, z4]);

z21 = 1/2[z13, point 6 of circ];
drawarrow (up--down) scaled 10 shifted z21 withcolor Blues 7 7;
drawarrow (up--down) scaled 10 shifted (xpart point 1/2 of trap, y21) withcolor Oranges 7 7;

label.bot(btex  $\displaystyle V_S = V_P = {1\over3}r^2\cdot2\pi r = {2\over3} \pi r^3$ etex, 
     point 1/2 of bbox currentpicture shifted 42 down);
\end{mplibcode}
$$
\vfill
\contrib{Sidney H.\@ Kung}

%-----------------------------------------
\chapter{Trigonometry, Calculus, \& Analytic Geometry}

\minitoc

\section{Sine of the sum}

\vfill
$$
\begin{mplibcode}
numeric x, y, z, r;
pair A, B, C, P;

x = 75; y = 46; x + y + z = 180;
r = 180;
A = r * dir (270 - z);
B = r * dir (270 + z);
C - A = whatever * dir x;
C - B = whatever * dir (180-y);
P = whatever[A,B]; C - P = whatever * up;

path am[];
am1 = fullcircle scaled 42 rotated angle (B-A) shifted A cutafter (A--C);
am2 = fullcircle scaled 42 rotated angle (C-B) shifted B cutafter (B--A);
am3 = fullcircle scaled 42 rotated angle (A-C) shifted C cutafter (C--B);
am4 = fullcircle scaled 42 rotated angle A cutafter (origin -- 1/2[A,B]);
forsuffixes $=1,2: draw am$ withcolor 3/4; endfor
forsuffixes $=3,4: draw am$ withcolor Reds 6 5; endfor

draw subpath (1,3) of unitsquare scaled 6 shifted 1/2[A,B] withcolor 3/4;
draw subpath (1,3) of unitsquare scaled 6 shifted P withcolor 3/4;

draw fullcircle scaled 2r withcolor Blues 7 6;
fill fullcircle scaled 2  withcolor Blues 7 6;

draw C--P dashed evenly withcolor 3/4;
draw A--B--C--A--origin--1/2[A,B];

label.urt("$\alpha$", point arctime 3/4 arclength am1 of am1 of am1);
label.lft("$\beta$", point arctime 1/2 arclength am2 of am2 of am2);
label.lrt("$\gamma$", point arctime 1/2 arclength am3 of am3 of am3);
label.llft("$\gamma$", point arctime 1/2 arclength am4 of am4 of am4);
label.urt("$a$", 1/2[B,C]);
label.ulft("$b$", 1/2[C, A]);
label.bot("$c$", 1/2[A, B]);
label.top("$c/2$", 5/16[A, B]);
label.top("$r$", 1/2 A);

label.top("$\sin(\alpha+\beta) = \sin\alpha\cos\beta + \cos\alpha\sin\beta$ for $\alpha+\beta < \pi$",
   point 5/2 of bbox currentpicture shifted 42 up);

label.bot("\vbox{\openup 6pt\halign{\hss # \hss\cr $c = a \cos\beta + b \cos\alpha$\cr $r=1/2$ \quad $\Longrightarrow$ \quad $\sin\gamma = {c/2\over1/2} = c$,\enspace $\sin\alpha=a$,\enspace $\sin\beta=b$\cr $\sin\bigl(\alpha+\beta\bigr) = \sin\bigl(\pi - (\alpha+\beta)\bigr) = \sin\gamma = \sin\alpha\cos\beta + \sin\beta\cos\alpha$\cr}}", point 1/2 of bbox currentpicture shifted 12 down);

\end{mplibcode}
$$
\vfill
\contrib{Sidney H.\@ Kung}

\section{Area and difference formulas}

\vfill
$$
\begin{mplibcode}
numeric x, y, a, b;
x = 56; y = 42; a = 120 / cosd(x); a * cosd(x) = b * cosd(y);

path t[]; 
t1 = origin -- b * dir y -- a * dir x -- cycle;
t2 = origin -- (xpart point 1 of t1, 0) -- point 1 of t1 -- cycle;

path a[];
a1 = fullcircle scaled 64 cutafter subpath (2,3) of t1;
a2 = fullcircle scaled 36 cutafter subpath (2,3) of t2;
a3 = fullcircle scaled 64 cutbefore subpath (2,3) of t2 cutafter subpath (2,3) of t1;  

picture P[];
P1 = image(
    fill t1 withcolor Blues 8 1;
    pair p; p = whatever[point 2 of t1, point 3 of t1];
    p - point 1 of t1 = whatever * ((point 3 of t1 - point2 of t1) rotated 90);
    drawoptions(withcolor Blues 7 3);
    draw subpath (1,3) of unitsquare scaled 5 
        rotated angle (point 3 of t1 - point 2 of t1) 
        shifted p withpen pencircle scaled 1/4;
    draw p -- point 1 of t1;
    drawoptions(withcolor Blues 7 5);
    pair q, r; 
    q = 7/8 point arctime 1/2 arclength a3 of a3 of a3; r = q + (8, -8);
    draw r .. q dashed withdots scaled 1/4 withpen pencircle scaled 1/4;
    draw a3; label("$\alpha-\beta$", r+(4,-4));
    drawoptions();
    draw t1; 
    label.ulft("$a$", point -5/8 of t1);
    label.lrt("$b$", point   5/8 of t1);
);
P2 = image(
    fill t1 withcolor Blues 8 1;
    fill t2 withcolor Oranges 8 1;
    drawoptions(withcolor Oranges 7 3);
    draw subpath (1,3) of unitsquare scaled 5 
        rotated 90
        shifted point 1 of t2
        withpen pencircle scaled 1/4;
    drawoptions(withcolor Reds 7 5);
    draw a1; label.rt("$\alpha$", point arctime 1/2 arclength a1 of a1 of a1);
    drawoptions(withcolor Oranges 7 5);
    draw a2; label.rt("$\beta$", point arctime 1/2 arclength a2 of a2 of a2);
    drawoptions();
    draw t1; draw subpath (0,2) of t2;
    label.ulft("$a$", point -9/16 of t1);
    label.lrt("$b$", point   9/16 of t1);
    label.rt(textext("\strut$a\sin\alpha$") rotated 90, 1/2[point 1 of t2, point 2 of t1]);
    label.bot("$b\cos\beta$", point 1/2 of t2);
);
P3 = image(
    fill t2 withcolor Oranges 8 1;
    drawoptions(withcolor Oranges 7 3);
    draw subpath (1,3) of unitsquare scaled 5 
        rotated 90
        shifted point 1 of t2
        withpen pencircle scaled 1/4;
    drawoptions(withcolor Oranges 7 5);
    draw a2; label.rt("$\beta$", point arctime 1/2 arclength a2 of a2 of a2);
    drawoptions();
    draw t2;
    label.ulft("$b$", point  -9/16 of t2);
    label.rt(textext("\strut$b\sin\beta$") rotated 90, point 3/2 of t2);
    label.bot("$a\cos\alpha$", point 1/2 of t2);
);
draw P1;
draw P2 shifted 124 right;
draw P3 shifted 280 right;
label("$=$", (128, 60));
label("$-$", (288, 60));

label.bot(btex \vbox{\openup 6pt\halign{\hfil $#$&${}=#$\hfil\cr
\frac12\cdot a\cdot b\sin\bigl(\alpha-\beta\bigr)&\frac12\cdot a\sin\alpha \cdot
b\cos\beta - \frac12\cdot a\cos\alpha\cdot b\sin\beta\cr
        \sin\bigl(\alpha-\beta\bigr)&\sin\alpha\cos\beta -
        \cos\alpha\sin\beta\cr}} etex, 
point 1/2 of bbox currentpicture shifted 13 down);
\end{mplibcode}
$$
\vfill
$$
\begin{mplibcode}
numeric a, b, alpha, beta;
alpha = 72; beta = 42; a = 120; a * sind(alpha) = b * cosd(beta);

path t[]; 
t1 = origin -- a * sind(alpha) * up -- a * cosd(alpha) * left -- cycle; 
t2 = origin -- b * sind(beta) * right -- b * cosd(beta) * up -- cycle;

pair p; p = whatever[point 1 of t1, point 2 of t1]; 
p - point 1 of t2 = whatever * ((point 2 of t1-point 1 of t1) rotated 90);

path a[];
a1 = fullcircle scaled 24 rotated 0 shifted point 2 of t1 cutafter subpath (1,2) of t1;
a2 = fullcircle scaled 24 rotated 270 shifted point 2 of t2 cutafter subpath (1,2) of t2;
a3 = fullcircle scaled 32 rotated (90+beta) shifted point 1 of t2 cutafter (p--point 1 of t2);

picture P[];

P1 = image(
    fill t1 withcolor Blues 8 1;
    fill t2 withcolor Oranges 8 1;

    drawoptions(withcolor Blues 7 3);
    draw subpath (1,3) of unitsquare scaled 5 
        rotated angle (point 2 of t1 - point 1 of t1) 
        shifted p withpen pencircle scaled 1/4;
    draw p -- point 1 of t2;
    drawoptions(withcolor Blues 7 5);
    draw a1; label("$\alpha$", point arctime 1/2 arclength a1 of a1 of a1 + (5,5));
    drawoptions(withcolor Oranges 7 5);
    draw a2; label("$\beta$", point arctime 1/2 arclength a2 of a2 of a2 + (2,-8));
    drawoptions(withcolor Reds 7 5);
    pair q, r; 
    q = 3/4[point 1 of t2, point arctime 1/2 arclength a3 of a3 of a3]; r = q + (9, 12);
    draw r .. q dashed withdots scaled 1/4 withpen pencircle scaled 1/4;
    draw a3; label("$\alpha-\beta$", r + 4 up);
    drawoptions();
    draw subpath (1,3) of unitsquare scaled 5 withcolor Oranges 7 3;
    draw t1; draw subpath (0, 2) of t2;
    label.ulft("$a$", point 3/2 of t1);
    label.urt("$b$", point 3/2 of t2);
);
P2 = image(
    fill t1 withcolor Blues 8 1;

    drawoptions(withcolor Blues 7 3);
    draw subpath (1,3) of unitsquare scaled 5 rotated 90 withpen pencircle scaled 1/4;
    drawoptions(withcolor Blues 7 5);
    draw a1; label("$\alpha$", point arctime 1/2 arclength a1 of a1 of a1 + (5,5));
    drawoptions();
    draw t1;
    label.rt  (textext("\strut$b\cos\beta$") rotated 90, point 1/2 of t1);
    label.ulft("$a$", point 3/2 of t1);
    label.bot ("\strut$a\cos\alpha$", point 5/2 of t1);
);
P3 = image(
    fill t2 withcolor Oranges 8 1;
    drawoptions(withcolor Oranges 7 5);
    draw a2; label("$\beta$", point arctime 1/2 arclength a2 of a2 of a2 + (2,-8));
    drawoptions();
    draw subpath (1,3) of unitsquare scaled 5 withcolor Oranges 7 3;
    draw t2;
    label.bot("\strut$b\sin\beta$", point 1/2 of t2);
    label.urt("$b$", point 3/2 of t2);
    label.lft(textext("\strut$a\sin\alpha$") rotated 90, point 5/2 of t2);
);

draw P1 shifted 200 left;
draw P2 shifted 20 left;
draw P3 shifted 60 right;
label("$=$", (-84, 64));
label("$+$", (20, 64));

label.bot(btex \vbox{\openup 6pt\halign{\hfil $#$&${}=#$\hfil\cr
\frac12 \cdot a \cdot b\cos\bigl(\alpha-\beta\bigr)&
\frac12 \cdot a\cos\alpha \cdot b\cos\beta +
\frac12 \cdot a\sin\alpha \cdot b\sin\beta\cr
        \cos\bigl(\alpha-\beta\bigr)&\cos\alpha\cos\beta +
        \sin\alpha\sin\beta\cr}} etex, 
point 1/2 of bbox currentpicture shifted 13 down);

\end{mplibcode}
$$
\vfill
\contrib{Sidney H.\@ Kung}

\section{The law of cosines I}

\vfill
$$
\begin{mplibcode}
numeric a, b, theta;
path A, B, C, Am, Bm;
a = 136; b = 9/16 a; theta = 40;

A = unitsquare scaled a rotated -90;
B = unitsquare scaled b rotated theta;
C = point 3 of A 
 -- point 1 of B rotatedabout(point 3 of A, -90)
 -- point 3 of A rotatedabout(point 1 of B, +90) 
 -- point 1 of B -- cycle;

z0 = whatever[point 0 of A, point 3 of A];
point 1 of B - z0 = whatever * up;
path arc; 
arc = quartercircle rotated 180 scaled 2 abs(point 1 of B - z0) 
      shifted point 1 of B
      cutbefore subpath (0,1) of B;

Am = unitsquare scaled -abs(z0 - point 3 of A) shifted point 3 of A;
Bm = unitsquare scaled abs(point 0 of arc - point 1 of B) rotated theta shifted point 0 of arc;

picture P[];

P1 = image(
draw subpath (1,3) of unitsquare scaled 6 shifted z0 withcolor 1/2;
draw z0 -- point 1 of B dashed evenly scaled 1/2;

draw arc dashed withdots scaled 1/4;

fill A withcolor Oranges 7 1;
fill B withcolor Oranges 7 1;
fill C withcolor Blues 7 1;

fill Am withcolor Blues 7 1;
fill Bm withcolor Blues 7 1;

draw A;
draw B;
draw C;
draw Am;
draw Bm;

label.bot ("$a$", point 3/2 of A);
label.ulft("$b$", point 5/2 of B);
label.urt ("$c$", point 3/2 of C);

label("$\theta$", 16 dir 1/2 theta);
label.bot("\strut$b\cost$", 1/2 z0);
draw thelabel.top("$b\sint$", origin) rotated theta shifted point 1/2 of Bm;
);
P2 = image(
    Bm := Bm rotatedabout(point 1 of B, 90-theta);
    draw subpath (1,3) of unitsquare scaled 6 shifted z0 withcolor 1/2;
    forsuffixes $=Am, Bm, C: fill $ withcolor Blues 7 1; draw $; endfor
    label("$\left(a - b \cost \right)^2$", center Am);
    label("$\left(b \sint \right)^2$", center Bm);
    label("$c^2$", center C);
);

draw P1;
draw P2 shifted 200 right;
drawarrow 160 right -- 190 right withpen pencircle scaled 2 withcolor Blues 7 5;
label.bot(btex \vbox{\openup 8pt\halign{\hfil $#$&${}=#$\hfil\cr
c^2 & \left(b\sint\right)^2 + \left(a - b\cost\right)^2\cr
&b^2\sin^2\theta + a^2 - 2ab\cost + b^2\cos^2\theta\cr
&a^2 + b^2\left(\sin^2\theta + \cos^2\theta\right) - 2ab\cost\cr
&a^2 + b^2 - 2ab\cost\cr}} etex, point 1/2 of bbox currentpicture shifted 32 down);
\end{mplibcode}
$$
\vfill
\contrib{Timothy A.\@ Sipka}

\section{The law of cosines II}

\vfill
$$
\begin{mplibcode}
path c; numeric a, b;  
c = fullcircle scaled 421;
a = 0.98; b = 2.718;
z0 = whatever[point a of c, point a+4 of c] = whatever[point 0 of c, point b of c];

fill center c -- point 0 of c -- z0 -- cycle withcolor Greens 7 1;

% mark the angles
draw unitsquare scaled 8 rotated angle (point 4 of c - point b of c) shifted point b of c withcolor 3/4;
draw halfcircle scaled 64 shifted point 0 of c cutbefore (point 0 of c -- point b of c) withcolor Reds 7 7;
label("$\theta$", 26 dir (180 - 1/4(180 - 45b)) shifted point 0 of c); 

draw point a of c -- point a + 4 of c;
draw point 4 of c -- point 0 of c -- point b of c -- cycle;
draw c withcolor Blues 7 7;
drawdot center c withpen pencircle scaled dotlabeldiam withcolor Blues 7 7;

label.bot("$a$", 1/2 point 0 of c);
label.bot("$a$", 1/2 point 4 of c);

draw thelabel.top("$a$", 1/2 point 4 of c) rotated 45a;
draw thelabel.top("$c$", 1/2 (abs(z0), 0)) rotated 45a;
draw thelabel.top("$a-c$", 1/2[(abs(z0), 0), point 0 of c]) rotated 45a;

draw thelabel.top("$2a\cost-b$", origin) rotated -1/2(180-45b) shifted 1/2[z0, point b of c];
draw thelabel.top("$b$",              origin) rotated -1/2(180-45b) shifted 1/2[z0, point 0 of c];

label.bot(btex \vbox{\openup 8pt\halign{\hfil $#$ \hfil\cr
\bigl(2a\cost - b\bigr) \cdot b = \bigl( a - c \bigr) \cdot \bigl(a + c \bigr)\cr
c^2 = a^2 + b^2 - 2ab\cost\cr}} etex, point 1/2 of bbox currentpicture shifted 42 down);

\end{mplibcode}
$$
\vfill
\contrib{Sidney H.\@ Kung}

\section{The law of cosines III (via Ptolemy's theorem)}

\vfill
$$
\begin{mplibcode}
path c; numeric a, b;  
c = fullcircle scaled 421;
a = 1/4; b = -7/8;

z0 = point a of c; z1 = point 4-a of c;
z3 = point b of c; z2 = point 4-b of c;

draw fullcircle scaled 42 rotated angle (z0-z3) shifted z3 cutafter (z2--z3) withcolor Reds 7 6;
draw fullcircle scaled 42 rotated angle (z3-z2) shifted z2 cutafter (z1--z2) withcolor Reds 7 6;

draw z0--z2--z3--z0--z1;
draw z2--z1--z3 dashed evenly;

draw c withcolor Blues 7 7;

label.top("$a+2b\cos\bigl(\pi-\theta\bigr)$", 1/2[z0, z1]);
label.bot("$a$", 1/2[z2, z3]);
label.lft("$b$", 1/2[z1, z2]);
label.rt ("$b$", 1/2[z3, z0]);
label.ulft("$c$", 3/4[z2, z0]);
label.urt ("$c$", 3/4[z3, z1]);

label("$\theta$", z3 + 8 (unitvector(z0-z3)+unitvector(z1-z3)));
label("$\theta$", z2 + 8 (unitvector(z0-z2)+unitvector(z1-z2)));

label.bot(btex \vbox{\openup 8pt\halign{\hfil $#$ \hfil\cr
c \cdot c = b \cdot b + \Bigl(a + 2b \cos\bigl(\pi-\theta\bigr)\Bigr) \cdot a\cr
c^2 = a^2 + b^2 - 2ab\cost\cr}} etex, point 1/2 of bbox currentpicture shifted 42 down);

\end{mplibcode}
$$
\vfill
\contrib{Sidney H.\@ Kung}

\section{The double-angle formulae}

\vfill
$$
\begin{mplibcode}
path h; pair A, B, C, D, O; numeric theta;

h = halfcircle scaled 420;

O = origin;
A = point 4 of h;
B = point 0 of h;
C = point 5/4 of h;
D = (xpart C, ypart A);

2theta = angle C;

draw unitsquare scaled 8 rotated angle (C-D) shifted D withcolor 3/4;
draw unitsquare scaled 8 rotated angle (A-C) shifted C withcolor 3/4;

draw A--C--B withcolor Reds 7 7;
draw O--C--D withcolor Reds 7 7;

drawoptions(withcolor Blues 7 6);
draw h; 
label.ulft("$x^2 + y^2 = 1$", point 3 of h);
drawoptions();

primarydef o through p = (1+o/arclength(p))[point 1 of p, point 0 of p] -- (1+o/arclength(p))[point 0 of p, point 1 of p] enddef;
drawarrow 16 through (A--B);
drawarrow 16 through (O--point 2 of h);

dotlabel.bot("$A$", A);
dotlabel.bot("$B$", B);
dotlabel.urt("$C \smash{\;\bigl(\cos2\theta, \sin2\theta\bigr)}$", C);
dotlabel.bot("$D$", D);
dotlabel.llft("$O$", O);

label("$\theta$", 28 dir 1/2 theta shifted A);
label("$2\theta$", 20 dir theta);

label("$x$", B shifted 24 right);
label("$y$", point 2 of h shifted 24 up);

draw thelabel.top("$2\cost$", origin) rotated theta      shifted 1/2[A, C];
draw thelabel.top("$2\sint$", origin) rotated (theta-90) shifted 1/2[B, C];

label.bot("$\triangle ACD \sim \triangle ABC$", point 1/2 of bbox currentpicture shifted 42 down);
path p; p = bbox currentpicture shifted 20 down;

label.bot(btex \vbox{\openup 8pt\halign{\hfil $#$\hfil\cr
CD \Big/ AC = BC \Big/ AB\cr
\sin 2\theta \big/ 2 \cost = 2 \sint \big/ 2\cr
\sin 2\theta = 2\sint \cost\cr}} etex, point 1/4 of p);

label.bot(btex \vbox{\openup 8pt\halign{\hfil $#$\hfil\cr
AD \Big/ AC = AC \Big/ AB\cr
\bigl(1 + \cos 2\theta \bigr) \big/ 2 \cost = 2 \cost \big/ 2\cr
\cos 2\theta = 2\cos^2\theta - 1\cr}} etex, point 3/4 of p);

\end{mplibcode}
$$
\vfill
\contrib{Roger B.\@ Nelsen}

\section{The half-angle tangent formulae}

\vfill
$$
\begin{mplibcode}
path h; pair A, B, C, D, O; numeric theta;

h = halfcircle scaled 420;

O = origin;
A = point 4 of h;
B = point 0 of h;
C = point 5/4 of h;
D = (xpart C, ypart A);

theta = angle C;

draw unitsquare scaled 8 rotated angle (C-D) shifted D withcolor 3/4;
draw unitsquare scaled 8 rotated angle (A-C) shifted C withcolor 3/4;

drawoptions(withcolor Reds 7 7);
draw A--C--B;
draw O--C--D;

drawoptions();
label("$\theta/2$", 38 dir 1/4 theta shifted A);
label("$\theta/2$", 42 dir (270 + 1/4 theta) shifted C);
label("$\theta$", 20 dir 1/2 theta);

drawoptions(withcolor Blues 7 6);
draw h; 
label.ulft("$x^2 + y^2 = 1$", point 3 of h);

drawoptions();
primarydef o through p = (1+o/arclength(p))[point 1 of p, point 0 of p] -- (1+o/arclength(p))[point 0 of p, point 1 of p] enddef;
drawarrow 16 through (A--B);
drawarrow 16 through (O--point 2 of h);

label("$x$", B shifted 24 right);
label("$y$", point 2 of h shifted 24 up);

label.bot("$1$", 1/2 A);
label.ulft("$1$", 1/2 C);
label.bot("$\cost$", 1/2 D);
label.bot("$1-\cost$", 1/2[B, D]);
draw thelabel.top("$\sint$", origin) rotated 90 shifted 1/2[C,D];

label.bot(btex $\displaystyle
\tan \theta\big/2 = {\sint\over 1+\cost} = {1-\cost\over\sint}
$ etex, point 1/2 of bbox currentpicture shifted 42 down);

\end{mplibcode}
$$
\vfill
\contrib{R.\@ J.\@ Walker}

\section{Mollweide's equation}

\vfill
$$
\begin{mplibcode}

z1 = 210 left;
z2 = 210 right;
z3 = 180 dir 113;

pair t; t = unitvector(z1-z3) + unitvector(z2-z3);
z4 - z3 = whatever * t;
z4 - z1 = whatever * t rotated 90;

z5 = whatever[z1,z4] = whatever[z2,z3];
z6 = whatever[z1,z4]; 
z7 = whatever[z1,z4];

z6 - z2 = whatever * (z3 - z4);
z7 - z2 = whatever * (z3 - z1);

draw subpath (1,3) of unitsquare scaled 8 rotated angle (z4-z1) shifted z4 withcolor 3/4;
draw subpath (1,3) of unitsquare scaled 8 rotated angle (z1-z6) shifted z6 withcolor 3/4;

draw z3--z4 dashed withdots scaled 1/2;
draw z2--z6 dashed withdots scaled 1/2;

drawoptions(withcolor Blues 8 7);
draw halfcircle scaled 48 shifted z1 cutafter (z1--z3);
label("$\alpha$", z1 + 32 dir 1/2 angle (z3-z1));

drawoptions(withcolor Greens 8 7);
draw reverse halfcircle scaled 48 shifted z2 cutafter (z2--z3);
label("$\beta$", z2 + 32 dir (90 + 1/2 angle (z3-z2)));

drawoptions(withcolor Oranges 8 7);
draw halfcircle scaled 48 rotated angle (z4-z3) shifted z3 cutafter (z2--z3);
draw halfcircle scaled 48 rotated angle (z7-z2) shifted z2 cutafter (z2--z6);
label("${\gamma\over2}$", z3 + 20 (unitvector(z4-z3) + unitvector(z2-z3)));

drawoptions(withcolor Reds 8 7);
picture a; a = image(
    for s=48,52:
        draw halfcircle scaled s rotated angle (z3-z2) shifted z5 cutafter (z5--z1);
    endfor
);
draw a;
draw a rotatedabout(z5, 180);
draw a rotatedabout(z5, 180) reflectedabout(z2,z6);
label("${\alpha+\beta\over2}$", z5 + 22 (unitvector(z1-z5) + unitvector(z3-z5)));

drawoptions(withcolor Purples 8 7);
draw halfcircle scaled 108 shifted z1 cutafter (z1--z4);
pair s, t; s = z1 + 58 dir 1/2 angle (z4-z1); t = s + (32, -18);
label.bot("$\alpha-\beta\over2$", t);
drawarrow t {up} .. {left} s withpen pencircle scaled 1/4;

drawoptions();
draw z1--z2--z3--cycle;
draw z1--z7--z2;

label.urt ("$a$", 1/2[z2, z3]);
label.ulft("$b$", 1/2[z3, z1]);
label.bot ("$c$", 1/2[z1, z2]);
label.lrt ("$a-b$", 1/2[z2, z7]);

label.top(btex $\displaystyle
(a-b)\cos{\gamma\over2} = c \sin\left(\alpha-\beta\over2\right)
$ etex, point 5/2 of bbox currentpicture shifted 42 up);

\end{mplibcode}
$$
\vfill
\contrib{H.\@ Arthur DeKleine}

\section{Tangent, cotangent, secant, and cosecant}

\vfill
$$
\begin{mplibcode}
numeric s, theta;
theta = 28;
s = 120;

path c, t;
c = fullcircle scaled 2s;

z0 = whatever * dir theta;
z1 = whatever * dir theta;
z2 = (x1, y0) = (xpart point 0 of c, ypart point 6 of c);

drawoptions(withcolor 3/4);
draw unitsquare scaled 6 rotated 0   shifted point 6 of c;
draw unitsquare scaled 6 rotated 90  shifted z2;
draw unitsquare scaled 6 rotated 180 shifted point 0 of c;
draw unitsquare scaled 6 rotated 270 shifted center c;

drawoptions(withcolor Blues 8 8);
draw c; 

drawoptions();
draw z0--z1--z2--cycle;
draw point 0 of c -- center c -- point 6 of c;

drawoptions(withcolor Blues 8 8);
drawdot center c withpen pencircle scaled dotlabeldiam;

drawoptions(withcolor Reds 8 7);
label("$\theta$", z0 + 24 dir 1/2 theta);
label("$\theta$", center c + 24 dir 1/2 theta);

drawoptions();
label.top("$1$", 1/2[point 0 of c, center c]);
label.lft("$1$", 1/2[point 6 of c, center c]);
label.bot("$1$", 1/2[point 6 of c, z2]);
label.rt ("$1$", 1/2[point 0 of c, z2]);
label.rt ("$\tan\theta$", 1/2[point 0 of c, z1]);
label.bot("$\cot\theta$", 1/2[point 6 of c, z0]);

picture p; 
p = thelabel.top("$\csc\theta$", origin);
unfill bbox p rotated theta shifted 1/2[z0, center c];
       draw p rotated theta shifted 1/2[z0, center c];
draw thelabel.top("$\sec\theta$", origin) rotated theta shifted 1/2[z1, center c];

label.bot(btex \vbox{\openup 8pt\halign{\hfil $#$ \hfil\cr
\tan^2\theta + 1 = \sec^2 \theta\cr
\cot^2\theta + 1 = \csc^2 \theta\cr
\left(\tan\theta + 1\right)^2 +  
\left(\cot\theta + 1\right)^2 = 
\left(\sec\theta + \csc\theta\right)^2\cr}} etex, point 1/2 of bbox currentpicture shifted 32 down);

label.bot(btex also $\displaystyle \tan\theta = {\tan\theta+1\over\cot\theta + 1}$ etex,
point 1/2 of bbox currentpicture shifted 24 down);


\end{mplibcode}
$$
\vfill
\contrib{William Romaine}

%------------------------
\section{Substitution to make a rational function of sine and cosine}
\vfill
$$
\begin{mplibcode}
numeric theta, u; theta = 60; u = 144;
z0 = u * left;
z1 = origin; 
x2 = x1; z2 = whatever * dir 1/2 theta shifted z0;
y3 = y1; z2 - z3 = whatever * (z2-z0) rotated 90;
z4 = z1 shifted (z2-z3);
z5 = z2 shifted (z2-z3);
x6 = x5; y6 = y1;
picture P[];
P0 = image(
    draw z0--z1--z2--cycle withpen pencircle scaled 1;
    label.bot("$1$", 1/2[z0, z1]);
    label.rt ("$z$", 1/2[z1, z2]);
    label("$\theta/2$", 32 dir 1/4 theta shifted z0 shifted 2 down) withcolor Reds 8 7;
);

P1 = image(
    draw unitsquare scaled 6 rotated 90 withcolor 1/2;
    draw P0;
);

P2 = image(
    draw unitsquare scaled 6 rotated 90 withcolor 1/2;
    draw z1--z3--z2; 
    draw P0;
    label.bot("$z^2$", 1/2[z1, z3]);
);

P3 = image(
    draw unitsquare scaled 6 shifted z1 withcolor 1/2;
    draw unitsquare scaled 6 shifted z4 withcolor 1/2;
    draw z5--z4--z2 -- cycle -- z0;
    draw z1--z3--z2; 
    draw P0;
    
    picture par_mark; par_mark = image(
        draw (up--down) scaled 3 rotated -10 shifted 3/4 left;
        draw (up--down) scaled 3 rotated -10 shifted 3/4 right;
    );
        
    draw par_mark rotated angle (z2-z3) shifted 1/2[z2,z3] withcolor Blues 8 7;
    draw par_mark rotated angle (z2-z3) shifted 1/2[z2,z5] withcolor Blues 8 7;

    label.ulft("$1+z^2$", 1/2[z0, z5]);
    label.bot("$z^2$", 1/2[z1, z3]);
    label.rt ("$z$", 1/2[z4, z5]);
    label("$\theta/2$", 36 dir 3/4 theta shifted z0) withcolor Reds 8 7;
);
P4 = image(
    draw unitsquare scaled 6 rotated 90 shifted z6 withcolor 1/2;
    draw z1--z2--z0--z3--z5;
    draw z0--z6--z5--cycle withpen pencircle scaled 1;
    picture t; t = thelabel("$\theta$", 18 dir 1/2 theta shifted z0);
    unfill bbox t; draw t withcolor Reds 8 7;
    label.ulft("$1+z^2$", 1/2[z0, z5]);
    label.bot("$z^2$", 1/2[z1, z3]);
    label.bot("$z^2$", 1/2[z1, z6]);
    label.bot("$1-z^2$", 1/2[z0, z6]);
    label.rt("$2z$", 1/2[z5, z6]);
);


P2 := P2 shifted (7/4u, 0);
P3 := P3 shifted (0,    -7/4u);
P4 := P4 shifted (7/4u, -7/4u);

draw P1;
draw P2;
draw P3;
draw P4;

drawoptions(withpen pencircle scaled 2 withcolor Blues 8 7);
interim linecap := butt;
interim linejoin := mitered;
interim bboxmargin := 16;
picture a; a = image(drawarrow (left--right) scaled 21);
drawoptions();

for i=1 upto 3:
    draw a rotated angle (center P[i+1] - center P[i]) shifted 1/2[center P[i], center P[i+1]];
endfor

label.bot(btex 
    $z=\tan\left(\theta/2\right) \mathbin{\:\Longrightarrow\:} 
    \hbox{$\displaystyle\sin\theta = {2z   \over 1 + z^2}$ \quad and\quad
          $\displaystyle\cos\theta = {1-z^2\over 1 + z^2}$}$ etex, 
          point 1/2 of bbox currentpicture
            shifted 42 down);

\end{mplibcode}
$$
\vfill
\contrib{Roger B.\@ Nelsen}
\end{document}
