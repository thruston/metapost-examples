\documentclass[oneside]{scrbook}
\usepackage{unicode-math}
\setmainfont{TeX Gyre Pagella}
\setmathfont{TeX Gyre Pagella Math}
\usepackage{graphicx}
\usepackage{mflogo}
\addtokomafont{section}{\clearpage}
\usepackage{luamplib}
\mplibtextextlabel{enable}
\everymplib{input colorbrewer-rgb;interim ahangle := 30;beginfig(0);}
\everyendmplib{
numeric wd, tw; 
wd = xpart (urcorner bbox currentpicture - llcorner bbox currentpicture);
tw = \mpdim{\textwidth};
if wd > tw: currentpicture := currentpicture scaled (tw/wd); fi
endfig;}
\usepackage{minitoc}
\mtcsetrules{*}{off}
%--------------------
\title{Proofs without words I\\[6pt]{\Large Exercises in \MP}}
\date{March 2021 —\\[4in]
\centerline{\begin{mplibcode}
    path t[], h;
    numeric r; r = -30;
    t0 = (for i=0 upto 2: up scaled 21 rotated 120i -- endfor cycle) rotated r;
    h  = (for i=0 upto 5: up scaled 34 rotated 60i -- endfor cycle) rotated r;
    t1 = subpath (0, 1) of t0 -- point 1 of h -- cycle;
    t2 = subpath (1, 2) of t0 -- point 3 of h -- cycle;
    t3 = subpath (2, 3) of t0 -- point 5 of h -- cycle;
    t4 = subpath (0, 1) of h -- point 0 of t0 -- cycle;
    t5 = subpath (1, 2) of h -- point 1 of t0 -- cycle;
    t6 = subpath (2, 3) of h -- point 1 of t0 -- cycle;
    t7 = subpath (3, 4) of h -- point 2 of t0 -- cycle;
    t8 = subpath (4, 5) of h -- point 2 of t0 -- cycle;
    t9 = subpath (5, 6) of h -- point 0 of t0 -- cycle;
    fill t0 withcolor Blues 7 2;
    fill t1 withcolor Blues 7 1;
    fill t2 withcolor Blues 7 3;
    fill t3 withcolor Blues 7 3;
    fill t4 withcolor Blues 7 1;
    fill t5 withcolor Blues 7 1;
    fill t6 withcolor Blues 7 2;
    fill t7 withcolor Blues 7 5;
    fill t8 withcolor Blues 7 5;
    fill t9 withcolor Blues 7 2;
    forsuffixes @=0, 4, 5, 6, 7, 8, 9: draw t@; endfor
\end{mplibcode}}}
\author{Toby Thurston}
\setcounter{secnumdepth}{-1}
\setcounter{tocdepth}{0}
\def\contrib#1{\rightline{— #1}}
\def\implies{\ensuremath\enspace\Longrightarrow\enspace}
\begin{document}
\dominitoc[n]
\maketitle
\tableofcontents
\chapter{Geometry and Algebra}

\minitoc

\section{The Pythagorean theorem I}

\vfill
$$
\begin{mplibcode}
path s, t; 
s = unitsquare shifted -(1/2, 1/2) scaled 144;
t = point 0 of s -- point 2/3 of s -- point -1/3 of s -- cycle;
picture P[];
P2 = image(
    for i=0 upto 3:
        fill t rotated 90i withcolor if odd i: Blues 7 2 else: Oranges 7 2 fi; 
        draw t rotated 90i;
    endfor
    draw s;
);
P1 = image(
    fill t withcolor Oranges 7 2; draw t;
    t := t rotatedabout(point 3/2 of t, 180);
    fill t withcolor Oranges 7 2; draw t;
    t := t shifted (point 0 of t - point 2 of t);
    t := t rotatedabout(point 2 of t, -90);
    fill t withcolor Blues 7 2; draw t;
    t := t rotatedabout(point 3/2 of t, 180);
    fill t withcolor Blues 7 2; draw t;
    draw s;
);
draw P1;
draw P2 shifted 200 right;
\end{mplibcode}
$$
\vfill
\contrib{adapted from the \textit{Chou pei san ching}}

\section{The Pythagorean theorem II}

\vfill
$$
\begin{mplibcode}
path s, t; 
s = fullcircle scaled 144;
t = (point 4 of s -- point 0 of s -- point sqrt(2) of s -- cycle) shifted point 6 of s;
picture P[];
P1 = image(
    for i=0 upto 3:
        fill t rotated 90i withcolor if odd i: Oranges 7 2 else: Blues 7 2 fi;
        draw t rotated 90i;
    endfor
);
P2 = image(
    t := t rotatedabout(point 0 of t, 180 - angle (point 2 of t - point 0 of t)) shifted (point 1 of t - point 0 of t); 
    fill t withcolor Blues 7 2; draw t;
    t := t rotatedabout(point 1/2 of t, 180); 
    fill t withcolor Blues 7 2; draw t;
    t := t rotatedabout(point 1 of t, -90); 
    fill t withcolor Oranges 7 2; draw t;
    t := t rotatedabout(point 1/2 of t, 180); 
    fill t withcolor Oranges 7 2; draw t;
    draw subpath (0, 2) of unitsquare scaled (abs(point 2 of t - point 1 of t) - abs(point 2 of t - point 0 of t)) shifted point 2 of t;
);

draw P1;
draw P2 shifted 180 right;
label.bot("\textit{Behold!}", point 1/2 of bbox currentpicture shifted 36 down);

    
\end{mplibcode}
$$

\vfill
\contrib{Bh\=askara (12th century)}

\section{The Pythagorean theorem III}

\vfill
$$
\begin{mplibcode}
path s, t, a, b, c; 
s = fullcircle scaled 72;
t = (point 4 of s -- point 0 of s -- point sqrt(6) of s -- cycle) shifted point 6 of s;
a = unitsquare scaled abs(point 2 of t - point 0 of t) rotated angle (point 2 of t - point 0 of t) shifted point 0 of t;
b = unitsquare scaled abs(point 1 of t - point 2 of t) rotated angle (point 1 of t - point 2 of t) shifted point 2 of t;
c = unitsquare scaled abs(point 0 of t - point 1 of t) rotated angle (point 0 of t - point 1 of t) shifted point 1 of t;
color v, w; v = Oranges 7 1; w = Greens 7 1;
picture P[];
P0 = image(
    draw a;
    draw b;
    draw c;
);
P1 = image(
    fill a withcolor v;
    fill b withcolor v;
    draw P0
);
z0 = whatever[point 2 of a, point 3 of a] = whatever[point 2 of b, point 3 of b];
z1 = whatever[z0, point 3 of a]; x1 = xpart point 0 of a;
z2 = whatever[z0, point 2 of b]; x2 = xpart point 1 of b;
path wedge; wedge = subpath (0,1) of a -- subpath (0, 1) of b -- z2 -- z0 -- z1 -- cycle;

P2 = image(
    draw point 2 of a -- z0 -- point 3 of b dashed evenly scaled 1/2;
    path a', b'; numeric t, u;
    t = angle (point 1 of a - point 0 of a);
    u = angle (point 1 of b - point 0 of b);
    a' = a shifted - point 0 of a rotated -t slanted 1/2 rotated t shifted point 0 of a;
    b' = b shifted - point 0 of b rotated -u slanted -1/3 rotated u shifted point 0 of b;
    fill a' withcolor 1/4[v,w]; draw a';
    fill b' withcolor 1/4[v,w]; draw b';
    draw P0
);
P3 = image(
    draw point 2 of a -- z0 -- point 3 of b dashed evenly scaled 1/2;
    fill wedge withcolor 1/2[v,w]; draw wedge; draw point 1 of a -- z0;
    draw P0
);
P4 = image(
    fill wedge shifted (point 0 of a - z1) withcolor 3/4[v,w]; 
    draw wedge shifted (point 0 of a - z1); 
    draw P0
);
P5 = image(
    fill c withcolor w;
    draw P0
);

draw P1;
draw P2 shifted (144,0);
draw P3 shifted (288,0);
draw P4 shifted (72, -200);
draw P5 shifted (216, -200);

    
\end{mplibcode}
$$
\vfill
\contrib{based on Euclid's proof}

\section{The Pythagorean theorem IV}


\vfill
$$
\begin{mplibcode}
path c, a, a', bq, bq'; numeric r; r = 59;
c = unitsquare shifted -(1/2, 1/2) scaled 160;
a = c scaled cosd(r) rotated r;
pair p, q;
p = whatever[point 0 of a, point 1 of a] = whatever[point 0 of c, point 1 of c];
q = whatever[point 0 of a, point 3 of a] = whatever[point 0 of c, point 3 of c];
bq = point 0 of c -- p -- point 0 of a -- q -- cycle;  

fill a withcolor Blues 7 2;
for i=0 upto 3:
    fill bq rotated 90i withcolor if odd i: Oranges 7 2 else: Purples 7 2 fi;
    draw bq rotated 90i;
endfor

a' = a shifted (point 3 of c - point 0 of a);
fill a' withcolor Blues 7 2; 
draw a';

bq' = bq rotated 180 shifted (point 1 of a' - point 2 of (bq rotated 180));
pair o; o = point 0 of bq';
for i=0 upto 3:
    fill bq' rotatedabout(o, 90i) withcolor if odd i: Oranges 7 2 else: Purples 7 2 fi;
    draw bq' rotatedabout(o, 90i);
endfor


\end{mplibcode}
$$
\vfill
\contrib{H.\@ E.\@ Dudeney (1917)}

\section{The Pythagorean theorem V}

\vfill
$$
\begin{mplibcode}
path t, t';
t = (origin -- 377 right -- 144 up -- cycle) scaled 3/4;
t' = t rotated -90 shifted (point 2 of t + point 1 of t rotated 90);

draw unitsquare scaled 8 withcolor 1/2;
draw unitsquare scaled 8 rotated -90 shifted point 0 of t' withcolor 1/2;
draw unitsquare scaled 8 rotated angle (point 1 of t - point 2 of t)
      shifted point 2 of t withcolor 1/2;

draw t;
draw t';
draw point 1 of t -- point 2 of t';

label.lft("$a$", point -1/2 of t);
label.bot("$b$", point 1/2 of t);
label.urt("$c$", point 3/2 of t);
label.top("$a$", point -1/2 of t');
label.lft("$b$", point 1/2 of t');
label.lrt("$c$", point 3/2 of t');

label.bot(btex \vbox{\openup 24pt\halign{\hfil $#$ \hfil\cr
A = 2 \cdot \frac12 ab + \frac12 c^2 = \frac12\left(a+b\right)^2\cr
c^2 = a^2 + b^2\cr}} etex, (xpart point 1 of t, ypart point 2 of t' - 12));


\end{mplibcode}
$$
\vfill
\contrib{James A.\@ Garfield (1876)}

\section{The Pythagorean theorem VI}

\vfill
$$
\begin{mplibcode}
numeric r;
r = 144;  z1 = r * dir 66;

draw unitsquare scaled 8 rotated 90 shifted (x1, 0) withcolor 1/2;
draw (left--right) scaled r withcolor Blues 7 7;
draw origin -- z1 -- (x1, 0) withcolor Blues 7 7;
draw fullcircle scaled 2r withcolor Reds 7 7;

label.top("$a$", (1/2 x1, 0));
label.rt("$b$", (x1, 1/2 y1));
label.ulft("$c$", 1/2 z1);
label.top("$c$", (-1/2 r, 0));
label.top("$c-a$", (1/2(r+x1), 0));

label.lft(btex \vbox{\openup 24pt\halign{\hfil $\displaystyle#$ \hfil\cr
{c+a\over b} = {b \over c-a} \cr
a^2 + b^2 = c^2\cr}} etex, point -1/2 of bbox currentpicture + 16 left);
\end{mplibcode}
$$
\vfill
\contrib{Michael Hardy}

\section{A Pythagorean theorem: $aa' = bb' + cc'$}

\vfill
$$
\begin{mplibcode}
path c; c = fullcircle scaled 377;
z0 = point 4 of c; z1 = point 0 of c; z2 = point 2.828 of c;
z3 = 5/16[z0, z1];
z4 = whatever [z1, z2]; 
z4 - z3 = whatever * (z2 - z0);
picture P;
P = image(
    draw unitsquare scaled 6 rotated angle (z0 - z2) shifted z2 withcolor 1/2; 
    draw unitsquare scaled 6 rotated angle (z3 - z4) shifted z4 withcolor 1/2;
    draw z3 -- z1 -- z2 -- z0 -- z3 -- z4;

    label.bot ("$a$", 1/2[z0, z1] shifted 10 down); label.top("$a'$", 7/16[z3, z1]);
    label.ulft("$b$", 1/2[z0, z2]); label.ulft("$b'$", 1/2[z3, z4]);
    label.urt ("$c$", 1/2[z1, z2]); label.llft("$c'$", 9/16[z1, z4]);

);

draw P shifted 200 up;
x5 = x4; y5 = 0;
draw unitsquare scaled 6 shifted z5 withcolor Blues 7 4;
draw z4--z5 withcolor Blues 7 4;
draw P;
label.bot("$\scriptstyle x$", 1/2[z3, z5]) withcolor Blues 7 6;
label.bot("$\scriptstyle y$", 1/2[z1, z5]) withcolor Blues 7 6;

label.bot(btex \vbox{\openup 8pt\halign{\hfil $\displaystyle # $\hfil\cr
{x\over b'} = {b\over a} \implies {x\over b} = {b'\over a} \implies ax = bb';\cr
{y\over c'} = {c\over a} \implies {y\over c} = {c'\over a} \implies ay = cc';\cr
\therefore\quad aa' = a\left(x+y\right) = bb' + cc'.\cr
}} etex, point 1/2 of bbox currentpicture shifted 24 down);
\end{mplibcode}
$$
\vfill
\contrib{Enzo R.\@ Gentile}

\section{The rolling circle squares itself}

\vfill
$$
\begin{mplibcode}
numeric r; r = 64;
numeric pi; pi = 3.141592653589793;  
path base, h, c, c', s;

base = (left--right) scaled 7/2r;
h = halfcircle rotated 180 scaled (pi * r + r);
c = fullcircle scaled 2r rotated 90 shifted point 0 of h shifted (0, r);
c' = fullcircle scaled 2r rotated 270 shifted point 4 of h shifted (-r, r);
s = unitsquare scaled (sqrt(pi) * r) rotated -90 shifted point 0 of c'; 

fill c withcolor Blues 7 1; 
fill s withcolor Blues 7 1;

draw base withcolor 1/2;
draw subpath (0, 4 + 1/45 angle point 1 of s) of h withcolor 1/2; 
draw subpath (4 + 1/45 angle point 1 of s, 4) of h withcolor Blues 7 3;
draw s;
draw point infinity of h -- point 2 of c' dashed evenly;

forsuffixes $=c, c': 
    draw point 0 of $ -- center $ -- point 2 of $ dashed evenly;
    draw $; drawdot point 0 of $ withpen pencircle scaled dotlabeldiam;
endfor

drawarrow subpath (5/4, 1/4) of fullcircle scaled (2r + 16) 
    shifted center c withcolor Reds 7 7;
\end{mplibcode}
$$
\vfill
\contrib{Thomas Elsner}

\section{On trisecting an angle}
\vfill
$$
\begin{mplibcode}
  picture link, pointer, pointer_groove;
  color metal, light_metal; 
  metal = (181, 166, 66)/256;
  light_metal = 3/4[metal,white];

  link = image(
      path a,b,a',b',c;
      a = fullcircle scaled 3;
      a' = a shifted (98,0);
      b = fullcircle scaled 5;
      b' = b shifted center a';
      c = subpath(2,6) of b -- 
          subpath(-2,2) of b' -- cycle;
          fill c withcolor light_metal; draw c;
          fill a withcolor metal; draw a;
          fill a' withcolor metal; draw a';
  );
  pointer = image(
      path a,b,c; numeric r;
      a = fullcircle scaled 18;
      b = fullcircle scaled 24;
      r = 1/3;
      c = subpath(r,8-r) of b --
          point 8-r of b shifted (10cm,0) --
          point 0   of b shifted (116mm,0) --
          point 8+r of b shifted (10cm,0) -- cycle;
      fill c withcolor light_metal; 
      fill subpath (9,11) of c -- cycle withcolor 7/8 white;
      draw point 9 of c -- point 11 of c; 
      draw c; 
      fill a withcolor white; draw a;
  );
  pointer_groove = image(
      draw pointer;
      path g;
      g = (halfcircle scaled 4 rotated 90 --
      halfcircle scaled 4 rotated 270 shifted (4cm,0) -- cycle) shifted (5cm,0);
      fill g withcolor 7/8[metal,white]; draw g;
  );
  draw pointer        rotated 42;
  draw pointer_groove rotated 28;
  draw pointer_groove rotated 14;
  draw pointer        rotated 0;
  z0 = 210 right rotated 14;  
  z1 = 120 right;
  numeric t; t = angle (z0-z1);
  draw link rotated t shifted z1 rotatedabout(z0,-34.5);
  draw link rotated t shifted z1 rotatedabout(z0,-34.5) rotated 14;
  draw link rotated t shifted z1;
  draw link rotated t shifted z1 rotated 14;

\end{mplibcode}
$$
\vfill
\contrib{Rufus Isaacs}

\section{Trisection in an infinite number of steps}
\vfill
$$
\begin{mplibcode}
numeric alpha, beta;
alpha = 144;
beta = 0;
for i=1 upto 9:
    beta := beta if odd i: + else: - fi alpha * (2 ** -i);
    path ray; 
    ray = origin -- (130 + 10i) * dir beta;
    draw ray withcolor 3/4;
    if i < 7:
        picture t; 
        t = thelabel("$\frac1{" & decimal (2**i) & "}$", origin) scaled (1 - i/8) shifted point 1 of ray; 
        unfill bbox t; draw t;
    fi
endfor
   
for i = 0 upto 3:
    draw origin -- 240 dir (i * alpha/3) 
        if i mod 3 > 0: dashed evenly fi
        withcolor Reds 6 6;
endfor
filldraw fullcircle scaled dotlabeldiam;
label.bot("$\displaystyle \frac13 = \frac12 - \frac14 +\frac18 - \frac1{16} + \cdots $",
    point 1/2 of bbox currentpicture shifted 24 down);
    
\end{mplibcode}
$$
\vfill
\contrib{Eric Kincanon}

\section{Trisection of a line segment}
\vfill
$$
\begin{mplibcode}
picture P[];
pair A, B, C, D, E, F;
A = origin;
B = 80 right;
C = B rotated 60;
D = C rotated 180;
E = 1/2 [B, D];
F = p[A,B] = q[E, C];

path ca, cb;
ca = fullcircle scaled 2 abs (A-B);
cb = ca rotated 180 shifted B;

P0 = image(
    drawoptions(withpen pencircle scaled 1/4 withcolor Blues 7 4);
    draw subpath 1/45(50, 70) of ca; draw subpath 1/45(50, 70) of cb;
    draw subpath -1/45(50, 70) of ca; draw subpath -1/45(50, 70) of cb;
    drawoptions();
);
P9 = image(
    draw A -- B;
    dotlabel.ulft("$A$", A);
    dotlabel.urt("$B$", B);
);
P1 = image(draw P0; draw P9);
P2 = image(
    draw P0;
    drawoptions(withpen pencircle scaled 1/4 withcolor Blues 7 4);
    draw subpath 1/45(230, 250) of ca; 
    drawoptions();
    draw C -- D withcolor 1/2;
    dotlabel.top("$C$", C);
    dotlabel.llft("$D$", D);
    draw P9;
);
P3 = image(
    draw B--D withcolor 1/2; 
    draw A -- C reflectedabout(A,B) dashed evenly withcolor 1/2;
    drawdot E withpen pencircle scaled dotlabeldiam;
    label("$E$", E-(2,12));
    draw P2;
);
P4 = image(
    draw B--C dashed evenly withcolor 1/2;
    draw C--E withcolor 1/2;
    draw P3;
    dotlabel.lrt("$F$", F);
);

draw P1;
draw P2 shifted (250, 0);
draw P3 shifted (0, -220);
draw P4 shifted (250, -220);

label.bot("$\overline{AF} = \frac13\cdot\overline{AB}$",  
    point 1/2 of bbox currentpicture shifted 36 down);
\end{mplibcode}
$$
\vfill
\contrib{Scott Cobel}
\end{document}
