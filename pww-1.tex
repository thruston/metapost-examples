\documentclass[oneside]{scrbook}
\usepackage{unicode-math}
\setmainfont{TeX Gyre Pagella}
\setmathfont{TeX Gyre Pagella Math}
\usepackage{graphicx}
\usepackage{mflogo}
\addtokomafont{section}{\clearpage}
\usepackage{luamplib}
\mplibtextextlabel{enable}
\everymplib{input colorbrewer-rgb;interim ahangle := 30;beginfig(0);}
\everyendmplib{
numeric wd, tw;
wd = xpart (urcorner bbox currentpicture - llcorner bbox currentpicture);
tw = \mpdim{\textwidth};
if wd > tw: currentpicture := currentpicture scaled (tw/wd); fi
endfig;}
\usepackage{minitoc}
\mtcsetrules{*}{off}
%--------------------
\title{Proofs without words I\\[6pt]{\Large Exercises in \MP}}
\date{March 2021 —\\[4in]
\centerline{\begin{mplibcode}
    path t[], h;
    numeric r; r = -30;
    t0 = (for i=0 upto 2: up scaled 21 rotated 120i -- endfor cycle) rotated r;
    h  = (for i=0 upto 5: up scaled 34 rotated 60i -- endfor cycle) rotated r;
    t1 = subpath (0, 1) of t0 -- point 1 of h -- cycle;
    t2 = subpath (1, 2) of t0 -- point 3 of h -- cycle;
    t3 = subpath (2, 3) of t0 -- point 5 of h -- cycle;
    t4 = subpath (0, 1) of h -- point 0 of t0 -- cycle;
    t5 = subpath (1, 2) of h -- point 1 of t0 -- cycle;
    t6 = subpath (2, 3) of h -- point 1 of t0 -- cycle;
    t7 = subpath (3, 4) of h -- point 2 of t0 -- cycle;
    t8 = subpath (4, 5) of h -- point 2 of t0 -- cycle;
    t9 = subpath (5, 6) of h -- point 0 of t0 -- cycle;
    fill t0 withcolor Blues 7 2;
    fill t1 withcolor Blues 7 1;
    fill t2 withcolor Blues 7 3;
    fill t3 withcolor Blues 7 3;
    fill t4 withcolor Blues 7 1;
    fill t5 withcolor Blues 7 1;
    fill t6 withcolor Blues 7 2;
    fill t7 withcolor Blues 7 5;
    fill t8 withcolor Blues 7 5;
    fill t9 withcolor Blues 7 2;
    forsuffixes @=0, 4, 5, 6, 7, 8, 9: draw t@; endfor
\end{mplibcode}}}
\author{Toby Thurston}
\setcounter{secnumdepth}{-1}
\setcounter{tocdepth}{0}
\def\contrib#1{\rightline{— #1}}
\def\implies{\ensuremath\enspace\Longrightarrow\enspace}
\begin{document}
\dominitoc[n]
\maketitle
\tableofcontents
\chapter{Geometry and Algebra}

\minitoc

\section{The Pythagorean theorem I}

\vfill
$$
\begin{mplibcode}
path s, t;
s = unitsquare shifted -(1/2, 1/2) scaled 144;
t = point 0 of s -- point 2/3 of s -- point -1/3 of s -- cycle;
picture P[];
P2 = image(
    for i=0 upto 3:
        fill t rotated 90i withcolor if odd i: Blues 7 2 else: Oranges 7 2 fi;
        draw t rotated 90i;
    endfor
    draw s;
);
P1 = image(
    fill t withcolor Oranges 7 2; draw t;
    t := t rotatedabout(point 3/2 of t, 180);
    fill t withcolor Oranges 7 2; draw t;
    t := t shifted (point 0 of t - point 2 of t);
    t := t rotatedabout(point 2 of t, -90);
    fill t withcolor Blues 7 2; draw t;
    t := t rotatedabout(point 3/2 of t, 180);
    fill t withcolor Blues 7 2; draw t;
    draw s;
);
draw P1;
draw P2 shifted 200 right;
\end{mplibcode}
$$
\vfill
\contrib{adapted from the \textit{Chou pei san ching}}

\section{The Pythagorean theorem II}

\vfill
$$
\begin{mplibcode}
path s, t;
s = fullcircle scaled 144;
t = (point 4 of s -- point 0 of s -- point sqrt(2) of s -- cycle) shifted point 6 of s;
picture P[];
P1 = image(
    for i=0 upto 3:
        fill t rotated 90i withcolor if odd i: Oranges 7 2 else: Blues 7 2 fi;
        draw t rotated 90i;
    endfor
);
P2 = image(
    t := t rotatedabout(point 0 of t, 180 - angle (point 2 of t - point 0 of t)) shifted (point 1 of t - point 0 of t);
    fill t withcolor Blues 7 2; draw t;
    t := t rotatedabout(point 1/2 of t, 180);
    fill t withcolor Blues 7 2; draw t;
    t := t rotatedabout(point 1 of t, -90);
    fill t withcolor Oranges 7 2; draw t;
    t := t rotatedabout(point 1/2 of t, 180);
    fill t withcolor Oranges 7 2; draw t;
    draw subpath (0, 2) of unitsquare scaled (abs(point 2 of t - point 1 of t) - abs(point 2 of t - point 0 of t)) shifted point 2 of t;
);

draw P1;
draw P2 shifted 180 right;
label.bot("\textit{Behold!}", point 1/2 of bbox currentpicture shifted 36 down);
\end{mplibcode}
$$

\vfill
\contrib{Bh\=askara (12th century)}

\section{The Pythagorean theorem III}

\vfill
$$
\begin{mplibcode}
path s, t, a, b, c;
s = fullcircle scaled 72;
t = (point 4 of s -- point 0 of s -- point sqrt(6) of s -- cycle) shifted point 6 of s;
a = unitsquare scaled abs(point 2 of t - point 0 of t) rotated angle (point 2 of t - point 0 of t) shifted point 0 of t;
b = unitsquare scaled abs(point 1 of t - point 2 of t) rotated angle (point 1 of t - point 2 of t) shifted point 2 of t;
c = unitsquare scaled abs(point 0 of t - point 1 of t) rotated angle (point 0 of t - point 1 of t) shifted point 1 of t;
color v, w; v = Oranges 7 1; w = Greens 7 1;
picture P[];
P0 = image(
    draw a;
    draw b;
    draw c;
);
P1 = image(
    fill a withcolor v;
    fill b withcolor v;
    draw P0
);
z0 = whatever[point 2 of a, point 3 of a] = whatever[point 2 of b, point 3 of b];
z1 = whatever[z0, point 3 of a]; x1 = xpart point 0 of a;
z2 = whatever[z0, point 2 of b]; x2 = xpart point 1 of b;
path wedge; wedge = subpath (0,1) of a -- subpath (0, 1) of b -- z2 -- z0 -- z1 -- cycle;

P2 = image(
    draw point 2 of a -- z0 -- point 3 of b dashed evenly scaled 1/2;
    path a', b'; numeric t, u;
    t = angle (point 1 of a - point 0 of a);
    u = angle (point 1 of b - point 0 of b);
    a' = a shifted - point 0 of a rotated -t slanted 1/2 rotated t shifted point 0 of a;
    b' = b shifted - point 0 of b rotated -u slanted -1/3 rotated u shifted point 0 of b;
    fill a' withcolor 1/4[v,w]; draw a';
    fill b' withcolor 1/4[v,w]; draw b';
    draw P0
);
P3 = image(
    draw point 2 of a -- z0 -- point 3 of b dashed evenly scaled 1/2;
    fill wedge withcolor 1/2[v,w]; draw wedge; draw point 1 of a -- z0;
    draw P0
);
P4 = image(
    fill wedge shifted (point 0 of a - z1) withcolor 3/4[v,w];
    draw wedge shifted (point 0 of a - z1);
    draw P0
);
P5 = image(
    fill c withcolor w;
    draw P0
);

draw P1;
draw P2 shifted (144,0);
draw P3 shifted (288,0);
draw P4 shifted (72, -200);
draw P5 shifted (216, -200);
\end{mplibcode}
$$
\vfill
\contrib{based on Euclid's proof}

\section{The Pythagorean theorem IV}

\vfill
$$
\begin{mplibcode}
path c, a, a', bq, bq'; numeric r; r = 59;
c = unitsquare shifted -(1/2, 1/2) scaled 160;
a = c scaled cosd(r) rotated r;
pair p, q;
p = whatever[point 0 of a, point 1 of a] = whatever[point 0 of c, point 1 of c];
q = whatever[point 0 of a, point 3 of a] = whatever[point 0 of c, point 3 of c];
bq = point 0 of c -- p -- point 0 of a -- q -- cycle;

fill a withcolor Blues 7 2;
for i=0 upto 3:
    fill bq rotated 90i withcolor if odd i: Oranges 7 2 else: Purples 7 2 fi;
    draw bq rotated 90i;
endfor

a' = a shifted (point 3 of c - point 0 of a);
fill a' withcolor Blues 7 2;
draw a';

bq' = bq rotated 180 shifted (point 1 of a' - point 2 of (bq rotated 180));
pair o; o = point 0 of bq';
for i=0 upto 3:
    fill bq' rotatedabout(o, 90i) withcolor if odd i: Oranges 7 2 else: Purples 7 2 fi;
    draw bq' rotatedabout(o, 90i);
endfor
\end{mplibcode}
$$
\vfill
\contrib{H.\@ E.\@ Dudeney (1917)}

\section{The Pythagorean theorem V}

\vfill
$$
\begin{mplibcode}
path t, t';
t = (origin -- 377 right -- 144 up -- cycle) scaled 3/4;
t' = t rotated -90 shifted (point 2 of t + point 1 of t rotated 90);

draw unitsquare scaled 8 withcolor 1/2;
draw unitsquare scaled 8 rotated -90 shifted point 0 of t' withcolor 1/2;
draw unitsquare scaled 8 rotated angle (point 1 of t - point 2 of t)
      shifted point 2 of t withcolor 1/2;

draw t;
draw t';
draw point 1 of t -- point 2 of t';

label.lft("$a$", point -1/2 of t);
label.bot("$b$", point 1/2 of t);
label.urt("$c$", point 3/2 of t);
label.top("$a$", point -1/2 of t');
label.lft("$b$", point 1/2 of t');
label.lrt("$c$", point 3/2 of t');

label.bot(btex \vbox{\openup 24pt\halign{\hfil $#$ \hfil\cr
A = 2 \cdot \frac12 ab + \frac12 c^2 = \frac12\left(a+b\right)^2\cr
c^2 = a^2 + b^2\cr}} etex, (xpart point 1 of t, ypart point 2 of t' - 12));
\end{mplibcode}
$$
\vfill
\contrib{James A.\@ Garfield (1876)}

\section{The Pythagorean theorem VI}

\vfill
$$
\begin{mplibcode}
numeric r;
r = 144;  z1 = r * dir 66;

draw unitsquare scaled 8 rotated 90 shifted (x1, 0) withcolor 1/2;
draw (left--right) scaled r withcolor Blues 7 7;
draw origin -- z1 -- (x1, 0) withcolor Blues 7 7;
draw fullcircle scaled 2r withcolor Reds 7 7;

label.top("$a$", (1/2 x1, 0));
label.rt("$b$", (x1, 1/2 y1));
label.ulft("$c$", 1/2 z1);
label.top("$c$", (-1/2 r, 0));
label.top("$c-a$", (1/2(r+x1), 0));

label.lft(btex \vbox{\openup 24pt\halign{\hfil $\displaystyle#$ \hfil\cr
{c+a\over b} = {b \over c-a} \cr
a^2 + b^2 = c^2\cr}} etex, point -1/2 of bbox currentpicture + 16 left);
\end{mplibcode}
$$
\vfill
\contrib{Michael Hardy}

\section{A Pythagorean theorem: $aa' = bb' + cc'$}

\vfill
$$
\begin{mplibcode}
path c; c = fullcircle scaled 377;
z0 = point 4 of c; z1 = point 0 of c; z2 = point 2.828 of c;
z3 = 5/16[z0, z1];
z4 = whatever [z1, z2];
z4 - z3 = whatever * (z2 - z0);
picture P;
P = image(
    draw unitsquare scaled 6 rotated angle (z0 - z2) shifted z2 withcolor 1/2;
    draw unitsquare scaled 6 rotated angle (z3 - z4) shifted z4 withcolor 1/2;
    draw z3 -- z1 -- z2 -- z0 -- z3 -- z4;

    label.bot ("$a$", 1/2[z0, z1] shifted 10 down); label.top("$a'$", 7/16[z3, z1]);
    label.ulft("$b$", 1/2[z0, z2]); label.ulft("$b'$", 1/2[z3, z4]);
    label.urt ("$c$", 1/2[z1, z2]); label.llft("$c'$", 9/16[z1, z4]);

);

draw P shifted 200 up;
x5 = x4; y5 = 0;
draw unitsquare scaled 6 shifted z5 withcolor Blues 7 4;
draw z4--z5 withcolor Blues 7 4;
draw P;
label.bot("$\scriptstyle x$", 1/2[z3, z5]) withcolor Blues 7 6;
label.bot("$\scriptstyle y$", 1/2[z1, z5]) withcolor Blues 7 6;

label.bot(btex \vbox{\openup 8pt\halign{\hfil $\displaystyle # $\hfil\cr
{x\over b'} = {b\over a} \implies {x\over b} = {b'\over a} \implies ax = bb';\cr
{y\over c'} = {c\over a} \implies {y\over c} = {c'\over a} \implies ay = cc';\cr
\therefore\quad aa' = a\left(x+y\right) = bb' + cc'.\cr
}} etex, point 1/2 of bbox currentpicture shifted 24 down);
\end{mplibcode}
$$
\vfill
\contrib{Enzo R.\@ Gentile}

\section{The rolling circle squares itself}

\vfill
$$
\begin{mplibcode}
numeric r; r = 64;
numeric pi; pi = 3.141592653589793;
path base, h, c, c', s;

base = (left--right) scaled 7/2r;
h = halfcircle rotated 180 scaled (pi * r + r);
c = fullcircle scaled 2r rotated 90 shifted point 0 of h shifted (0, r);
c' = fullcircle scaled 2r rotated 270 shifted point 4 of h shifted (-r, r);
s = unitsquare scaled (sqrt(pi) * r) rotated -90 shifted point 0 of c';

fill c withcolor Blues 7 1;
fill s withcolor Blues 7 1;

draw base withcolor 1/2;
draw subpath (0, 4 + 1/45 angle point 1 of s) of h withcolor 1/2;
draw subpath (4 + 1/45 angle point 1 of s, 4) of h withcolor Blues 7 3;
draw s;
draw point infinity of h -- point 2 of c' dashed evenly;

forsuffixes $=c, c':
    draw point 0 of $ -- center $ -- point 2 of $ dashed evenly;
    draw $; drawdot point 0 of $ withpen pencircle scaled dotlabeldiam;
endfor

drawarrow subpath (5/4, 1/4) of fullcircle scaled (2r + 16)
    shifted center c withcolor Reds 7 7;
\end{mplibcode}
$$
\vfill
\contrib{Thomas Elsner}

\section{On trisecting an angle}
\vfill
$$
\begin{mplibcode}
  picture link, pointer, pointer_groove;
  color metal, light_metal;
  metal = 1/256 (181, 166, 66);
  light_metal = 3/4[metal, white];

  link = image(
      path a, b, a', b', c;
      a = fullcircle scaled 3; a' = a shifted (98,0);
      b = fullcircle scaled 5; b' = b shifted center a';
      c = subpath(2,6) of b -- subpath(-2,2) of b' -- cycle;
      fill c withcolor light_metal; draw c;
      fill a withcolor metal; draw a;
      fill a' withcolor metal; draw a';
  );
  pointer = image(
      path a, b, c; numeric r;
      a = fullcircle scaled 18;
      b = fullcircle scaled 24;
      r = 1/3;
      c = subpath(r,8-r) of b --
          point 8-r of b shifted (10cm,0) --
          point 0   of b shifted (116mm,0) --
          point 8+r of b shifted (10cm,0) -- cycle;
      fill c withcolor light_metal;
      fill subpath (9,11) of c -- cycle withcolor 7/8 white;
      draw point 9 of c -- point 11 of c;
      draw c;
      fill a withcolor white; draw a;
  );
  pointer_groove = image(
      draw pointer;
      path g;
      g = (halfcircle scaled 4 rotated 90 --
           halfcircle scaled 4 rotated 270 shifted (4cm,0) -- 
           cycle) shifted (5cm,0);
      fill g withcolor 7/8[metal,white]; draw g;
  );
  draw pointer        rotated 42;
  draw pointer_groove rotated 28;
  draw pointer_groove rotated 14;
  draw pointer        rotated 0;
  z0 = 210 right rotated 14;
  z1 = 120 right;
  numeric t; t = angle (z0-z1);
  draw link rotated t shifted z1 rotatedabout(z0,-34.5);
  draw link rotated t shifted z1 rotatedabout(z0,-34.5) rotated 14;
  draw link rotated t shifted z1;
  draw link rotated t shifted z1 rotated 14;
\end{mplibcode}
$$
\vfill
\contrib{Rufus Isaacs}

\section{Trisection in an infinite number of steps}
\vfill
$$
\begin{mplibcode}
numeric alpha, beta;
alpha = 144;
beta = 0;
for i=1 upto 9:
    beta := beta if odd i: + else: - fi alpha * (2 ** -i);
    path ray;
    ray = origin -- (130 + 10i) * dir beta;
    draw ray withcolor 3/4;
    if i < 7:
        picture t;
        t = thelabel("$\frac1{" & decimal (2**i) & "}$", origin) scaled (1 - i/8) shifted point 1 of ray;
        unfill bbox t; draw t;
    fi
endfor

for i = 0 upto 3:
    draw origin -- 240 dir (i * alpha/3)
        if i mod 3 > 0: dashed evenly fi
        withcolor Reds 6 6;
endfor
filldraw fullcircle scaled dotlabeldiam;
label.bot("$\displaystyle \frac13 = \frac12 - \frac14 +\frac18 - \frac1{16} + \cdots $",
    point 1/2 of bbox currentpicture shifted 24 down);

\end{mplibcode}
$$
\vfill
\contrib{Eric Kincanon}

\section{Trisection of a line segment}
\vfill
$$
\begin{mplibcode}
picture P[];
pair A, B, C, D, E, F;
A = origin;
B = 80 right;
C = B rotated 60;
D = C rotated 180;
E = 1/2 [B, D];
F = p[A,B] = q[E, C];

path ca, cb;
ca = fullcircle scaled 2 abs (A-B);
cb = ca rotated 180 shifted B;

P0 = image(
    drawoptions(withpen pencircle scaled 1/4 withcolor Blues 7 4);
    draw subpath 1/45(50, 70) of ca; draw subpath 1/45(50, 70) of cb;
    draw subpath -1/45(50, 70) of ca; draw subpath -1/45(50, 70) of cb;
    drawoptions();
);
P9 = image(
    draw A -- B;
    dotlabel.ulft("$A$", A);
    dotlabel.urt("$B$", B);
);
P1 = image(draw P0; draw P9);
P2 = image(
    draw P0;
    drawoptions(withpen pencircle scaled 1/4 withcolor Blues 7 4);
    draw subpath 1/45(230, 250) of ca;
    drawoptions();
    draw C -- D withcolor 1/2;
    dotlabel.top("$C$", C);
    dotlabel.llft("$D$", D);
    draw P9;
);
P3 = image(
    draw B--D withcolor 1/2;
    draw A -- C reflectedabout(A,B) dashed evenly withcolor 1/2;
    drawdot E withpen pencircle scaled dotlabeldiam;
    label("$E$", E-(2,12));
    draw P2;
);
P4 = image(
    draw B--C dashed evenly withcolor 1/2;
    draw C--E withcolor 1/2;
    draw P3;
    dotlabel.lrt("$F$", F);
);

draw P1;
draw P2 shifted (250, 0);
draw P3 shifted (0, -220);
draw P4 shifted (250, -220);

label.bot("$\overline{AF} = \frac13\cdot\overline{AB}$",
    point 1/2 of bbox currentpicture shifted 36 down);
\end{mplibcode}
$$
\vfill
\contrib{Scott Cobel}

\section{The vertex angles of a star sum to 180°}
\vfill
$$
\begin{mplibcode}
z3 = -z5 = 120 left;
z1 = 180 dir 81;
z2 = 250 dir 130;
z4 =  90 dir -100;

z6 = z5 + 72 right;
z7 = whatever [z2, z5] = whatever [z1, z4];
z8 = whatever [z3, z5] = whatever [z1, z4];
y9 = y1; z9 - z5 = whatever * (z1 - z4);

path star;
star = z3 -- z5 -- z2 -- z4 -- z1 -- cycle;
draw star withcolor Blues 7 7;
draw z6 -- z5 -- z9 dashed evenly withcolor Blues 7 5;

def angle_point(expr a, b, c, r) = b + r * (unitvector(a-b) + unitvector(c-b)) enddef;

label("$α$", angle_point(z3, z1, z4, 12));
label("$β$", angle_point(z5, z2, z4, 16));
label("$γ$", angle_point(z5, z3, z1, 10));
label("$δ$", angle_point(z1, z4, z2, 12));
label("$ε$", angle_point(z2, z5, z3, 12));

label("$α+γ$", angle_point(z1, z8, z5, 16) + 8 down);
label("$α+γ$", angle_point(z9, z5, z6, 16) + 8 down);
label("$β+δ$", angle_point(z5, z7, z4, 18) + 1 up);
label("$β+δ$", angle_point(z2, z5, z9, 18) + 2 down);
\end{mplibcode}
$$
\vfill
\contrib{Fouad Nakhli}

\section{Viviani's theorem I}
\vfill
$$
\begin{mplibcode}
pair A', B', C', A, B, C, D, E, F, G, P, Q, R;
A' = origin;
B' = 300 right;
C' = B' rotated 60;
P = 3/8[B', C'];
xpart Q = xpart C';
ypart Q = ypart E = ypart P;
D = whatever[A', B']; xpart D = xpart P;
E = whatever[A', C'];
R = whatever[A', C']; R - P = whatever * (A' - C') rotated 90;
G = whatever[C', Q] = whatever [R, P];
A = whatever[A', B']; G-A = whatever * (C' - A');
B - B' = A - A' = C - C';
F = whatever[B, C]; F-P = whatever * (B-C) rotated 90;

def right_angle_mark(expr a, b, s) =
    subpath (1,3) of unitsquare scaled s rotated angle(b-a) shifted a
enddef;

drawoptions(withcolor 1/2);
draw right_angle_mark(D, B, 6);
draw right_angle_mark(F, P, 6);
draw right_angle_mark(G, P, 6);
draw right_angle_mark(Q, P, 6);
draw right_angle_mark(R, A', 6);
draw E--P;
draw A'--B'--C'--cycle;
drawoptions();

draw P--F withcolor Reds 7 7;
draw R--G--Q withcolor 1/2[Reds 7 7, white];
draw G--P dashed evenly scaled 3/4 withcolor Greens 7 7;
draw G--C' dashed evenly scaled 3/4 withcolor 1/2[Greens 7 7, white];

draw P--D dashed withdots scaled 1/4 withcolor Blues 7 7;
draw A--B--C--cycle;

drawdot P withpen pencircle scaled dotlabeldiam;

forsuffixes $=A, A', B, B', D, Q: label.bot("\strut$" & str $ & "$", $); endfor
forsuffixes $=C, C': label.top("$" & str $ & "$", $); endfor
label.urt("$F$", F);
label("$G$", G + 10 dir 192);

label.top(btex \vbox{\halign{\hss #\hss\cr
The perpendiculars to the sides from a point on\cr
the boundary or within an equilateral triangle\cr
add up to the height of the triangle.\cr
}} etex, point 5/2 of bbox currentpicture shifted 42 up);

label("\textit{This shows a particular example, with $C'GQ$ collinear, rather than the general case}",
point 1/2 of bbox currentpicture shifted 42 down);
\end{mplibcode}
$$
\vfill
\contrib{Samuel Wolf}

\section{Viviani's theorem II}
\vfill
$$
\begin{mplibcode}

def distance(expr a, b, c) = abs ypart ((a-b) rotated -angle (c-b)) enddef;

pair a, b, c, p;
a = 89 up;
b = a rotated 120;
c = b rotated 120;
p = 21 dir 42;

numeric h[];
h0 = distance(a, b, c);
h1 = distance(p, a, b);
h2 = distance(p, b, c);
h3 = distance(p, c, a);

path t[];
t0 = a--b--c--cycle;
t1 = t0 rotated -120 shifted -point 2 of t0 scaled (h1/h0) shifted p;
t2 = t0 shifted -point 0 of t0 scaled (h2/h0) shifted p;
t3 = t0 rotated +120 shifted -point 1 of t0 scaled (h3/h0) shifted p;

z0 = 1/3[1/2[point 2 of t1, point 1 of t3], point 0 of t0];
z1 = 2/3[point 0 of t1, point 3/2 of t1];

color s[];
s1 = Reds 7 2; s2 = Oranges 7 2; s3 = Blues 7 2;
picture p[];
forsuffixes $=1,2,3: p$ = image(fill t$ withcolor s$; draw t$ --point 3/2 of t$); endfor

picture P[];
P1 = image(draw p1; draw p2; draw p3; draw t0;);
P2 = image(
    path cor;
    cor = reverse fullcircle rotated 90 scaled 4/3 h1 scaled 15/16 shifted z1;
    drawarrow subpath 1/45(20, 100) of cor withcolor Reds 7 7;
    drawarrow subpath 1/45(140, 220) of cor withcolor Reds 7 7;
    drawarrow subpath 1/45(260, 340) of cor withcolor Reds 7 7;
    draw p1 rotatedabout(z1, -120); draw p2; draw p3; draw t0;);
P3 = image(
    path cor;
    cor = reverse fullcircle rotated 90 scaled 4/3 (h1+h3) scaled 7/8 shifted z0;
    drawarrow subpath 1/45(20, 100) of cor withcolor Reds 7 7;
    drawarrow subpath 1/45(140, 220) of cor withcolor Reds 7 7;
    drawarrow subpath 1/45(260, 340) of cor withcolor Reds 7 7;
    draw point 2 of t1 -- point 1 of t3 dashed withdots scaled 1/2;
    draw p2;
    draw p1 rotatedabout(z1, -120) rotatedabout(z0, -120);
    draw p3 rotatedabout(z0, -120);
    draw t0;);

draw P1 shifted 160 up;
draw P2 shifted 90 left;
draw P3 shifted 90 right;
label.top(btex \vbox{\halign{\hss #\hss\cr
The perpendiculars to the sides from a point on\cr
the boundary or within an equilateral triangle\cr
add up to the height of the triangle.\cr
}} etex, point 5/2 of bbox currentpicture shifted 42 up);
\end{mplibcode}
$$
\vfill
\contrib{Ken-Ichiroh Kawasaki}

\section{A theorem about right angles}
\vfill
$$
\begin{mplibcode}

path s, t;
s = unitsquare shifted -(1/2,1/2) scaled 210;
t = subpath (3, 2) of s -- point 1.732 of fullcircle scaled 210
  shifted point 5/2 of s -- cycle;

fill t withcolor Blues 7 1;
fill s withcolor Oranges 7 1;
for i=0 upto 3: draw t rotated 90i; endfor;
draw point 2 of t -- point 2 of t rotated 180;

label.top(btex \vbox{\halign{\hss #\hss\cr
The internal bisector of the right angle of a right\cr
triangle bisects the square on the hypotenuse\cr
}} etex, point 5/2 of bbox currentpicture shifted 42 up);
\end{mplibcode}
$$
\vfill
\contrib{Roland H.\@ Eddy}

\section{Area and the projection theorem of a right triangle}
\vfill
$$
\begin{mplibcode}
def angle_arc(expr a, o, b, r) = fullcircle scaled 2r rotated angle (a-o) shifted o cutafter (o--b) enddef;

path c; c = fullcircle scaled 180;
pair A, B, C, D, E, F;
A = point 4 of c;
B = point 0 of c;
C = point 1.7 of c;
D = (xpart C, ypart A);
E = C rotatedabout(D, -90);
F = B rotatedabout(D, -90);

color r, b, g; r = Reds 7 1; g = Greens 7 1; b = Blues 7 1;
picture P[];
P1 = image(
    fill A--B--C--cycle withcolor r;
    draw unitsquare scaled 6 rotated angle (A-C) shifted C withcolor 3/4 r;
    draw unitsquare scaled 6 rotated angle (C-D) shifted D withcolor 3/4 r;
    draw angle_arc(D, A, C, 12) withpen pencircle scaled 1 withcolor Reds 7 5;
    draw angle_arc(D, C, B, 12) withpen pencircle scaled 1 withcolor Reds 7 5;
    draw D--C--B--A--C;
    label.bot("$A$", A);
    label.top("$C$", C);
    label.bot("$D$", D);
    label.bot("$B$", B);
);

P2 = image(
    fill A--D--C--cycle withcolor r;
    fill A--D--F--cycle withcolor g;
    fill F--D--E--cycle withcolor r;
    fill C--D--E--cycle withcolor b;
    draw unitsquare scaled 6 rotated angle (C-D) shifted D withcolor 3/4 r;
    draw angle_arc(D, A, C, 12) withpen pencircle scaled 1 withcolor Reds 7 5;
    draw angle_arc(D, E, F, 12) withpen pencircle scaled 1 withcolor Reds 7 5;
    drawarrow subpath (7/4, 1/4) of quartercircle scaled 42 shifted D withcolor Blues 6 6;
    draw A--F--E--C--A--E;
    draw C--F;
    label.lft ("$A$", A);
    label.top ("$C$", C);
    label.llft("$D$", D);
    label.rt  ("$E$", E);
    label.bot ("$F$", F);
);

P3 = image(
    fill A--D--C--cycle withcolor r;
    fill C--D--E--cycle withcolor b;
    z3 = whatever[A,C]; z3 - E = whatever * (A-C) rotated 90;
    begingroup; interim ahangle := 180;
    drawarrow E--z3 dashed evenly scaled 3/4 withpen pencircle scaled 1/4;
    label.urt("$h$", 1/4[z3, E]);
    endgroup;
    draw A--E--C--A;
    draw C--D;
    label.bot("$A$", A);
    label.top("$C$", C);
    label.bot("$D$", D);
    label.bot("$E$", E);
);

P4 = image(
    fill A--D--C--cycle withcolor r;
    fill A--D--F--cycle withcolor g;
    z4 = whatever[A,C]; z4 - F = whatever * (A-C) rotated 90;
    draw z4--F dashed evenly scaled 3/4 withpen pencircle scaled 1/4;
    label.urt("$h$", 1/4[z4, F]);
    draw A--C--F--A--D;
    label.lft ("$A$", A);
    label.urt ("$C$", C);
    label.rt  ("$D$", D);
    label.lrt ("$F$", F);
);

P5 = image(
    fill C--D--E--cycle withcolor b;
    draw D--E--C--D;
    label.top("$C$", C);
    label.bot("$D$", D);
    label.bot("$E$", E);
);

P6 = image(
    fill A--D--F--cycle withcolor g;
    draw A--F--D--A;
    label.lft ("$A$", A);
    label.rt  ("$D$", D);
    label.lrt ("$F$", F);
);

draw P1 shifted 120 left;
draw P2 shifted 120 right;
numeric y; y = -190;
draw P3 shifted (-120, y); label("${}={}$", (12, y+16)); draw P4 shifted (+120, y);
y := y - 112;
draw P5 shifted (-120, y-abs(D-B)); label("${}={}$", (12, y-28)); draw P6 shifted (+120, y);
label("$CD^2 = AD\cdot DB$", point 1/2 of bbox currentpicture shifted 42 down);
\end{mplibcode}
$$
\vfill
\contrib{Sidney H.\@ Kung}

\section{Chords and tangents of equal length}
\vfill
$$
\begin{mplibcode}
def angle_arc(expr a, o, b, r) = fullcircle scaled 2r rotated angle (a-o) shifted o cutafter (o--b) enddef;

path C[]; pair O, P, Q, R;
C1 = fullcircle scaled 280; O = point 0 of C1;
C2 = fullcircle scaled 200 shifted O;

numeric t, u;
(t, u) = C1 intersectiontimes C2;
P = point t of C1;
Q = point 8-t of C1;
z0 = whatever[P, P + direction t of C1]; y0 = ypart point 6 of C1;
R = C2 intersectionpoint (z0--P);

draw center C1 -- P -- point 4 of C1 -- O withcolor 7/8;

forsuffixes $=P, Q, R: 
    draw O -- $ withcolor Blues 7 6; 
endfor
draw 5/4[Q, P] -- 5/4[P, Q] withcolor Reds 7 6;
draw 5/4[P, R] -- 5/4[R, P] withcolor Reds 7 6;

draw angle_arc(O, Q, P, 30);
draw angle_arc(O, Q, P, 28);
draw angle_arc(O, P, R, 30);
draw angle_arc(O, P, R, 28);

draw C1 withcolor 1/2;
draw C2 withcolor 1/2;;

dotlabel.urt("$O$", O);
dotlabel.urt("\strut $P$", P);
dotlabel.lrt("\strut $Q$", Q);
dotlabel.rt("$\;R$", R);

label.top(btex \vbox{\openup6pt\halign{\hss #\hss\cr
If circle $C_1$ passes through the center $O$ of circle $C_2$, the length\cr
of the common chord $\overline{PQ}$ is equal to the tangent segment $\overline{PR}$.\cr
}} etex, point 5/2 of bbox currentpicture shifted 42 up);
\end{mplibcode}
$$
\vfill
\contrib{Roland H.\@ Eddy}

\section{Completing the square}
\vfill
$$
\begin{mplibcode}
path xx, ax, hax, haha;
numeric x, a;
x = 89; a = 34;
xx   = unitsquare shifted 1/2 left scaled x shifted 12 up;
ax   = unitsquare shifted 1/2 left xscaled x yscaled -a shifted 12 down;
hax  = unitsquare shifted 1/2 left xscaled x yscaled -1/2 a shifted 12 down;
haha = unitsquare scaled 1/2 a rotated -90 shifted point 1 of xx shifted (8, -8);

picture P[];
P1 = image(
    fill xx withcolor Oranges 7 1; draw xx; 
    label.top("$x$", point 5/2 of xx); 
    label.lft("$x$", point 7/2 of xx); 
    label("${}+{}$", origin);
    fill ax withcolor Blues 7 2; draw ax;
    label.lft("$a$", point 7/2 of ax);
);

P2 = image(
    fill xx withcolor Oranges 7 1; draw xx; 
    label("${}+{}$", origin);
    for i=0, 1:
        fill hax shifted (0, -24i) withcolor Blues 7 2; 
        draw hax shifted (0, -24i);
    endfor
);

P3 = image(
    fill xx withcolor Oranges 7 1; draw xx; 
    hax := hax shifted (point 0 of xx - point 0 of hax);
    fill hax withcolor Blues 7 2; draw hax;
    hax := hax shifted - point 0 of hax rotated 90 shifted point 1 of xx;
    fill hax withcolor Blues 7 2; draw hax;

    fill haha withcolor Blues 7 1;
    draw haha dashed withdots scaled 1/4;
);


draw P1 shifted 144 left;
label("$=$", (-72, 16));
draw P2;
label("$=$", (72, 16));
draw P3 shifted 144 right;

label.top("$x^2 + ax = \left(x + a/2\right)^2 - \left(a/2\right)^2$", 
point 5/2 of bbox currentpicture shifted 42 up);
\end{mplibcode}
$$
\vfill
\contrib{Charles D.\@ Gallant}

\section{Algebraic areas I}
\vfill
$$
\begin{mplibcode}
numeric a, b;
a = 89; b = 21;
picture P[];
P1 = image(
    fill unitsquare xscaled a yscaled b shifted (0, a) withcolor Greens 7 1;
    fill unitsquare xscaled b yscaled a shifted (a, 0) withcolor Greens 7 1;
    draw (a, 0) -- (a, a+b) dashed withdots scaled 1/4;
    draw (0, a) -- (a+b, a) dashed withdots scaled 1/4;
    draw (a-b, a) -- (a-b, a+b) dashed withdots scaled 1/4;
    draw (a, a-b) -- (a+b, a-b) dashed withdots scaled 1/4;
    draw unitsquare scaled (a+b);
    label.bot("\strut $a$", (1/2a, 0));
    label.bot("\strut $b$", (a+1/2b, 0));
    label.lft("$a$", (0, 1/2a));
    label.lft("$b$", (0, a+1/2b));
);
P2 = image(
    draw unitsquare scaled (a-b);
    label.bot("\strut $a-b$", 1/2(a-b, 0));
);
P3 = image(
    draw unitsquare scaled a;
    draw unitsquare scaled b shifted (a,a);
    label.bot("\strut $a$", (1/2a, 0));
    label("$b$", (a + 1/2b, a + 1/2b));
);
P4 = image(
    fill unitsquare scaled a withcolor Greens 7 1;
    fill unitsquare scaled b shifted (a,a) withcolor Greens 7 1;
    fill unitsquare scaled (a-b) withcolor background;
    draw (a-b, a) -- (a-b, a-b) -- (a, a-b) dashed withdots scaled 1/4;
    draw (0, a-b) -- (a-b, a-b) -- (a-b, 0);
    draw unitsquare scaled a;
    draw unitsquare scaled b shifted (a,a);
    label.bot("\strut $a-b$", 1/2(a-b, 0));
    label.bot("\strut $b$", (a-1/2b, 0));
    label("$b$", (a + 1/2b, a + 1/2b));
);


numeric x, y;
draw P1;

y = 3/4 (a-b);
x = a + b + 14;
label("$+$", (x, y));

x := x + 14;
draw P2 shifted (x, 0);

x := x + a - b + 16;
label("$=$", (x, y));

x := x + 16;
draw P3 shifted (x, 0);

x := x + a + 14;
label("$+$", (x, y));

x := x + 14;
draw P4 shifted (x, 0);

label.top("$\left(a+b\right)^2 + \left(a-b\right)^2 = 2\left(a^2 + b^2\right)$", 
point 5/2 of bbox currentpicture shifted 42 up);
\end{mplibcode}
$$
\vfill
\contrib{Shirley Wakin}

\section{Algebraic areas II}
\vfill
$$
\begin{mplibcode}
input arrow_label

numeric a, b, c;
a = 80; 2b = a; 2c = b;

def make_box(expr p, shade) = image(fill p withcolor shade; draw p) enddef;


path s[];
s1 = unitsquare scaled (a-b-c); 
s2 = unitsquare scaled (2c); 
s3 = unitsquare scaled (a-b+c);
s4 = unitsquare scaled (2b);
s5 = unitsquare scaled (a+b-c); 
s6 = unitsquare xscaled (a+b-c) yscaled (a-b+c);
s7 = unitsquare xscaled (a-b+c) yscaled (a+b-c);

picture t[];
t1 = make_box(s1, Reds 7 2);
t2 = make_box(s2, Oranges 7 2);
t3 = make_box(s3, YlGn 7 2);
t4 = make_box(s4, Greens 7 2);
t5 = make_box(s5, Blues 7 2);
t6 = make_box(s6, Purples 7 2);
t7 = make_box(s7, Purples 7 2);

picture P[];
P1 = image(
    draw t5;
    draw t7 shifted point 1 of s5;
    draw t1 shifted (point 2 of s5 - point 3 of s1);
    draw t3 shifted point 2 of s5;
    draw t6 shifted point 3 of s5;

    draw t4 shifted (2a + 30, 0);
    draw t2 shifted (2a + 30 + 2b + 30, 0);

    arrow_label(origin, 2a * right, "$2a$", 9);
    label.bot("\strut$2b$", (2a + 30 + b, 0));
    label.bot("\strut$2c$", (2a + 30 + 2b + 30 + c, 0));
);
P2 = image(
    draw t4;
    draw t7 shifted point 1 of s4;
    draw t6 shifted point 3 of s4;
    draw t2 shifted ((1,1) scaled (a+b-c));

    draw t5 shifted (a + b + c + 20, 0);
    draw t3 shifted (2a + 2b + 40, 0);
    draw t1 shifted (3a + b + c + 60, 0);

    arrow_label(origin, 2b * up, "$2b$", -12);
    arrow_label((0, a+b+c), (a+b+c, a+b+c), "\strut$a+b+c$", -12);
    label.rt("$2c$", (a+b+c, a+b));
    label.top("$a+b-c$", (3/2a + 3/2b + 1/2c + 20, a + b - c + 4));
    label.top("$a-b+c$", (5/2a + 3/2b + 1/2c + 40, a - b + c + 4));
    label.top("$a-b-c$", (7/2a + 1/2b + 1/2c + 60, a - b - c + 4));
);

label.top(P1, origin);
label.top(P2, (0, 2a+2b));


label.top(btex $\left(a+b+c\right)^2 
          + \left(a+b-c\right)^2 
          + \left(a-b+c\right)^2 
          + \left(a-b-c\right)^2 
          = \left(2a\right)^2 
          + \left(2b\right)^2
          + \left(2c\right)^2$ etex, 
point 5/2 of bbox currentpicture shifted 42 up);
\end{mplibcode}
$$
\vfill
\contrib{Sam Pooley and K.\@ Add Drude}
\end{document}
