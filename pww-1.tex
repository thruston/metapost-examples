\documentclass[oneside]{scrbook}
\usepackage{unicode-math}
\setmainfont{TeX Gyre Pagella}
\setmathfont{TeX Gyre Pagella Math}
\usepackage{graphicx}
\usepackage{mflogo}
\addtokomafont{section}{\clearpage}
\usepackage{luamplib}
\mplibtextextlabel{enable}
\everymplib{input colorbrewer-rgb;interim ahangle := 30;beginfig(0);}\everyendmplib{endfig;}
\usepackage{minitoc}
\mtcsetrules{*}{off}
%--------------------
\title{Proofs without words I\\[6pt]{\Large Exercises in \MP}}
\date{March 2021 —\\[4in]
\centerline{\begin{mplibcode}
    path t[], h;
    numeric r; r = -30;
    t0 = (for i=0 upto 2: up scaled 21 rotated 120i -- endfor cycle) rotated r;
    h  = (for i=0 upto 5: up scaled 34 rotated 60i -- endfor cycle) rotated r;
    t1 = subpath (0, 1) of t0 -- point 1 of h -- cycle;
    t2 = subpath (1, 2) of t0 -- point 3 of h -- cycle;
    t3 = subpath (2, 3) of t0 -- point 5 of h -- cycle;
    t4 = subpath (0, 1) of h -- point 0 of t0 -- cycle;
    t5 = subpath (1, 2) of h -- point 1 of t0 -- cycle;
    t6 = subpath (2, 3) of h -- point 1 of t0 -- cycle;
    t7 = subpath (3, 4) of h -- point 2 of t0 -- cycle;
    t8 = subpath (4, 5) of h -- point 2 of t0 -- cycle;
    t9 = subpath (5, 6) of h -- point 0 of t0 -- cycle;
    fill t0 withcolor Blues 7 2;
    fill t1 withcolor Blues 7 1;
    fill t2 withcolor Blues 7 3;
    fill t3 withcolor Blues 7 3;
    fill t4 withcolor Blues 7 1;
    fill t5 withcolor Blues 7 1;
    fill t6 withcolor Blues 7 2;
    fill t7 withcolor Blues 7 5;
    fill t8 withcolor Blues 7 5;
    fill t9 withcolor Blues 7 2;
    forsuffixes @=0, 4, 5, 6, 7, 8, 9: draw t@; endfor
\end{mplibcode}}}
\author{Toby Thurston}
\setcounter{secnumdepth}{-1}
\setcounter{tocdepth}{0}
\def\contrib#1{\rightline{— #1}}
\begin{document}
\dominitoc[n]
\maketitle
\tableofcontents
\chapter{Geometry and Algebra}

\minitoc

\section{The Pythagorean theorem I}

\vfill
$$
\begin{mplibcode}
path s, t; 
s = unitsquare shifted -(1/2, 1/2) scaled 144;
t = point 0 of s -- point 2/3 of s -- point -1/3 of s -- cycle;
picture P[];
P2 = image(
    for i=0 upto 3:
        fill t rotated 90i withcolor if odd i: Blues 7 2 else: Oranges 7 2 fi; 
        draw t rotated 90i;
    endfor
    draw s;
);
P1 = image(
    fill t withcolor Oranges 7 2; draw t;
    t := t rotatedabout(point 3/2 of t, 180);
    fill t withcolor Oranges 7 2; draw t;
    t := t shifted (point 0 of t - point 2 of t);
    t := t rotatedabout(point 2 of t, -90);
    fill t withcolor Blues 7 2; draw t;
    t := t rotatedabout(point 3/2 of t, 180);
    fill t withcolor Blues 7 2; draw t;
    draw s;
);
draw P1;
draw P2 shifted 200 right;
\end{mplibcode}
$$
\vfill
\contrib{adapted from the \textit{Chou pei san ching}}

\section{The Pythagorean Theorem II}

\vfill
$$
\begin{mplibcode}
path s, t; 
s = fullcircle scaled 144;
t = (point 4 of s -- point 0 of s -- point sqrt(2) of s -- cycle) shifted point 6 of s;
picture P[];
P1 = image(
    for i=0 upto 3:
        fill t rotated 90i withcolor if odd i: Oranges 7 2 else: Blues 7 2 fi;
        draw t rotated 90i;
    endfor
);
P2 = image(
    t := t rotatedabout(point 0 of t, 180 - angle (point 2 of t - point 0 of t)) shifted (point 1 of t - point 0 of t); 
    fill t withcolor Blues 7 2; draw t;
    t := t rotatedabout(point 1/2 of t, 180); 
    fill t withcolor Blues 7 2; draw t;
    t := t rotatedabout(point 1 of t, -90); 
    fill t withcolor Oranges 7 2; draw t;
    t := t rotatedabout(point 1/2 of t, 180); 
    fill t withcolor Oranges 7 2; draw t;
    draw subpath (0, 2) of unitsquare scaled (abs(point 2 of t - point 1 of t) - abs(point 2 of t - point 0 of t)) shifted point 2 of t;
);

draw P1;
draw P2 shifted 180 right;
label.bot("\textit{Behold!}", point 1/2 of bbox currentpicture shifted 36 down);

    
\end{mplibcode}
$$

\vfill
\contrib{Bh\=askara (12th century)}

\section{The Pythagorean Theorem III}

\vfill
$$
\begin{mplibcode}
path s, t, a, b, c; 
s = fullcircle scaled 72;
t = (point 4 of s -- point 0 of s -- point sqrt(6) of s -- cycle) shifted point 6 of s;
a = unitsquare scaled abs(point 2 of t - point 0 of t) rotated angle (point 2 of t - point 0 of t) shifted point 0 of t;
b = unitsquare scaled abs(point 1 of t - point 2 of t) rotated angle (point 1 of t - point 2 of t) shifted point 2 of t;
c = unitsquare scaled abs(point 0 of t - point 1 of t) rotated angle (point 0 of t - point 1 of t) shifted point 1 of t;
color v, w; v = Oranges 7 1; w = Greens 7 1;
picture P[];
P0 = image(
    draw a;
    draw b;
    draw c;
);
P1 = image(
    fill a withcolor v;
    fill b withcolor v;
    draw P0
);
z0 = whatever[point 2 of a, point 3 of a] = whatever[point 2 of b, point 3 of b];
z1 = whatever[z0, point 3 of a]; x1 = xpart point 0 of a;
z2 = whatever[z0, point 2 of b]; x2 = xpart point 1 of b;
path wedge; wedge = subpath (0,1) of a -- subpath (0, 1) of b -- z2 -- z0 -- z1 -- cycle;

P2 = image(
    draw point 2 of a -- z0 -- point 3 of b dashed evenly scaled 1/2;
    path a', b'; numeric t, u;
    t = angle (point 1 of a - point 0 of a);
    u = angle (point 1 of b - point 0 of b);
    a' = a shifted - point 0 of a rotated -t slanted 1/2 rotated t shifted point 0 of a;
    b' = b shifted - point 0 of b rotated -u slanted -1/3 rotated u shifted point 0 of b;
    fill a' withcolor 1/4[v,w]; draw a';
    fill b' withcolor 1/4[v,w]; draw b';
    draw P0
);
P3 = image(
    draw point 2 of a -- z0 -- point 3 of b dashed evenly scaled 1/2;
    fill wedge withcolor 1/2[v,w]; draw wedge; draw point 1 of a -- z0;
    draw P0
);
P4 = image(
    fill wedge shifted (point 0 of a - z1) withcolor 3/4[v,w]; 
    draw wedge shifted (point 0 of a - z1); 
    draw P0
);
P5 = image(
    fill c withcolor w;
    draw P0
);

draw P1;
draw P2 shifted (144,0);
draw P3 shifted (288,0);
draw P4 shifted (72, -200);
draw P5 shifted (216, -200);

    
\end{mplibcode}
$$
\vfill
\contrib{based on Euclid's proof}

\end{document}
